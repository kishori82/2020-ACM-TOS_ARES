\remove{We proceed by first introducing some definitions and notation, then by presenting some properties that are satisfied 
by the reconfiguration service in any execution, and then we show that given these properties our algorithm satisfies 
the safety (atomicity) conditions. 
%Due to space limitations the proofs of the main statements are omitted and can be found in \cite{ARES:Arxiv:2018}.
}

\remove{
\myparagraph{Notations and definitions.}
For a server $s$, we use the notation $\atT{s.var}{\state}$ to refer to the value of the state variable $var$, in $s$, at a state $\state$ of an  execution $\EX$. 
If server  $s$ crashes at a state $\state_f$ in an execution $\EX$ then $\atT{s.var}{\state}\triangleq\atT{s.var}{\state_f}$ for any state variable $var$ and for 
any state $\state$ that appears after $\state_f$ in $\EX$. 
%refers to the value of $v$ at $s$ at the state just before crashes. In other words, $\atT{s.v}{T}  \triangleq \atT{s.v}{\hat{T}}$, where $\hat{T}$ is the latest point in the execution, such that, $(a)$ $\hat{T} \leq T$ and $(b)$ $s$ is non-faulty.
 }
 
 \remove{
\begin{definition}[Tag of a configuration]  Let  $c \in \mathcal{C}$ be a configuration, $\state$ be a state in some execution $\EX$ then 
we define the tag of $c$ at state $\state$ as  
$ \atT{tag(c)}{\state} \triangleq \min_{Q \in \quorums{c}} \max_{s \in Q}~\atT{(s.tag}{\state}).$
We  drop the suffix $|_\state$, and simply denote as $tag(c)$,  when the state  is clear from the context.
\end{definition}

\begin{definition}
Let $\sigma$ be any point in an execution of \ares{} and suppose we use the notation $\cvec{\pr}{\state}$ for $ \atT{\pr.cseq}{\state}$,  i.e., the $cseq$ variable at process $p$ at the state $\state$. %be the value of a configuration sequence vector at a process $\pr$ at some state  $\st$ in an execution $\EX$. 
Then we define as $ \mu(\cvec{\pr}{\state})  \triangleq  \max\{ i : \cvec{\pr}{\state}[i].status = F\}$ 
and $ \nu(\cvec{\pr}{\state}) \triangleq |\cvec{\pr}{\state}|$, where $|\cvec{\pr}{\state}|$ is the length of the  configuration vector 
$\cvec{\pr}{\state}$. % that are not equal to $\bot$.  
\end{definition}

\begin{definition} [Prefix order]
Let $\mathbf{x}$ and $\mathbf{y}$ be any two configuration sequences. We say that $\mathbf{x}$ is a prefix of $\mathbf{y}$, denoted by 
$\mathbf{x} \preceq_p  \mathbf{y}$, if $\config{\mathbf{x}[j]}=\config{\mathbf{y}[j]}$, for all $j$ such that $\mathbf{x}[j]\neq\bot$.
\end{definition}

%\input{ssec-correct-recon.tex}
\nn{Next we analyze the properties  satisfied by  ~\ares{}. In brief, the reconfiguration algorithm
preserves three properties in any execution: 
$(i)$ \myemph{Configuration Uniqueness}, 
$(ii)$ \myemph{Configuration 
Prefix}, and 
$(iii)$ \myemph{Configuration Progress}. Configuration Uniqueness ensures that no two different 
configurations are assigned to the same index in a configuration sequence. Configuration
prefix ensures that any \act{read-config} action 
returns an extension of the configuration sequence returned by any previous \act{read-config} action. 
Finally, the configuration progress focuses on the status of the configurations in a sequence, 
and ensures that a \act{read-config} action observes at least as recent
\myemph{finalized} configuration as any preceding \act{read-config} action. The following theorem presents formally those 
properties.}

%The first lemma shows that any two configuration sequences have the same configuration identifiers
%in the same indexes. 
%
%\begin{lemma}[Configuration Uniqueness]
%	\label{lem:unique}
%	
%	%Let $\state_1$ and $\state_2$ be any two states of an execution $\EX$ of the algorithm,
%	%and $\pr, q$ two participating processes. 
%	%be the state after the response action of an operation $\op_1$ from process $p$,
%	%and $\state_2$ be the state after the first $\act{read-config}$ call of an operation $\op_2$ from $q$.
%	For any processes $\pr, q\in \idSet$ and any states $\state_1, \state_2$ in an execution $\EX$, it must hold that 
%	$\config{\cvec{\pr}{\state_1}[i]}=\config{\cvec{q}{\state_2}[i]}$,  $\forall i$ s.t. 
%	$\config{\cvec{\pr}{\state_1}[i]},\config{\cvec{q}{\state_2}[i]}\neq \bot$.
%\end{lemma}
%
%We can now move to an important lemma that shows that any \act{read-config} action 
%returns an extension of the configuration sequence returned by any previous \act{read-config} action. 
%First, we show that the last finalized configuration observed by any \act{read-config} action is at least as 
%recent as the finalized configuration observed by any subsequent \act{read-config} action. 
%%Using this lemma we will then show that when 
%
%\begin{lemma}[Configuration Prefix]
%	\label{lem:prefix}
%	Let $\op_1$ and $\op_2$ two 
%	%read/write/install operations 
%	completed \act{read-config} actions invoked by processes $\pr_1, \pr_2\in\idSet$ 
%	respectively, such that $\op_1\bef\op_2$ in an execution $\EX$. Let $\state_1$ be the state after the response 
%	step of $\op_1$ and $\state_2$ the state after the response step 
%	%termination of the first $\act{read-config}$ 
%	of $\op_2$. Then 
%	$\cvec{\pr_1}{\state_1}\preceq_p\cvec{\pr_2}{\state_2}$.
%\end{lemma}
%
%Thus far we focused on the configuration member of each element in $cseq$. As operations do get in account
%the \myemph{status} of a configuration, i.e. $P$ or $F$, in the next lemma we will examine the relationship of 
%the last finalized configuration as detected by two operations. First we present a lemma that shows the 
%monotonicity of the finalized configurations.
%
%
%\begin{lemma}  [Configuration Progress]
%	\label{lem:finalconf}
%	Let $\op_1$ and $\op_2$ two 
%	%read/write/install operations 
%	completed \act{read-config} actions invoked by processes $\pr_1, \pr_2\in\idSet$ 
%	respectively, such that $\op_1\bef\op_2$ in an execution $\EX$. 
%	Let $\state_1$ be the state after the response 
%	step of $\op_1$ and $\state_2$ the state after the response step 
%	%after the completion of $\op_1$ and $\state_2$ the state after the termination of the first $\act{read-config}$ 
%	of $\op_2$. Then 
%	$\mu(\cvec{\pr_1}{\state_1})\leq\mu(\cvec{\pr_2}{\state_2})$.
%\end{lemma}


\begin{theorem}[Reconfiguration Properties]
	Let $\op_1$ and $\op_2$ be two 
	%read/write/install operations 
	completed \act{read-config} actions invoked by processes $\pr_1, \pr_2\in\idSet$ 
	respectively, such that $\op_1\bef\op_2$ in an execution $\EX$ of ~\ares. 
	Let $\state_1$ be the state after the response 
	step of $\op_1$ and $\state_2$ the state after the response step 
	%after the completion of $\op_1$ and $\state_2$ the state after the termination of the first $\act{read-config}$ 
	of $\op_2$.
	%$\cvec{\pr_1}{\state_1}$ and $\cvec{\pr_2}{\state_2}$ at two states $\state_1$ and $\state_2$ 
	%in an execution $\EX$ of the algorithm  such that $\state_1$ appears before $\state_2$ in $\EX$. 
	Then the following holds: 
%	\begin{enumerate}
	%	\item [$(a)$] 
	$(a)$	$\cvec{\pr_2}{\state_2}[i].cfg = \cvec{\pr_1}{\state_1}[i].cfg$,  for $ 1 \leq i \leq \nu(\cvec{\pr_1}{\state_1})$,
	%	\item [$(b)$]
		%$(b)$  
	$(b)$	$\cvec{\pr_1}{\state_1}  \preceq_p \cvec{\pr_2}{\state_2}$, and
	%	\item [$(c)$] 
		%$(c)$  
	$(c)$	$\mu(\cvec{\pr_1}{\state_1}) \leq \mu(\cvec{\pr_2}{\state_2})$
		%; and 
		%\item [$(d)$]  
		%\nn{????$(d)$  $\cvec{\pr_2}{\state_2}[i]   = \cvec{\pr_1}{\state_1}[i]$,  for  $ 1 \leq i \leq \mu(\cvec{\pr_1}{\state_1})$????} 
%	\end{enumerate}
\end{theorem}
}

%\input{ssec-correct-safety_v2.tex}

%\subsection{\ares{} Liveness}
%\label{ssec:liveness}
%\input{ssec-correct-liveness.tex}

%The propagation of the information of the distributed object in \ares{} is achieved using the $\act{get-tag}$, $\act{get-data}$, 
%and $\act{put-data}$ actions.  
Given the properties satisfied by the reconfiguration algorithm of \ares{} 
and assuming that the DAP used satisfy properties $C1$ and $C2$, as presented
in Section \ref{ssec:dap}, then  we have the following result. 

\begin{theorem}[Atomicity]
    Consider  any execution of \ares{} such that in every configuration used   the DAP primitives  satisfy  the DAP consistency properties then ~\ares{} satisfy atomicity property.
	%, given that the 
	%$\act{get-data}$, $\act{get-tag}$, and $\act{put-data}$ primitives used satisfy properties
	%\textbf{C1} and \textbf{C2} of Definition \ref{def:consistency}.
\end{theorem}

%In  \ares{},  each configuration 
%may implement the DAPs in a different way as stated below:

\begin{remark}
	Algorithm \ares{} satisfies atomicity even when the DAP primitives used in two 
	different configurations $c_1$ and $c_2$ are not the same, given that the $c_i.\act{get-tag}$,
	$c_i.\act{get-data}$, and the $c_i.\act{put-data}$ primitives 
	used in each $c_i$ satisfy properties \textbf{C1} and \textbf{C2} of Definition \ref{def:consistency}.  
\end{remark}



































