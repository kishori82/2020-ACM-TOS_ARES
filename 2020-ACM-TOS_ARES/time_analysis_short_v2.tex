\myparagraph{\ares{} Correctness.}
The correctness of \ares{} highly depends on the way the configuration 
sequence is constructed at each client process. Let $\cvec{p}{\state}$ 
denote the configuration sequence $cseq$ at process $p$ in a state $\state$ and 
$\mu(\cvec{\pr}{\state})$ the index of the last finalized configuration in $\cvec{p}{\state}$. 
Then the following properties are preserved by the reconfiguration service:

\begin{theorem}
	Let $\op_1$ and $\op_2$ two 
	%read/write/install operations 
	completed \act{read-config} actions invoked by processes $\pr_1, \pr_2\in\idSet$ 
	respectively, such that $\op_1\bef\op_2$ in an execution $\EX$. 
	Let $\state_1$ the state after the response 
	step of $\op_1$ and $\state_2$ the state after the response step 
	%after the completion of $\op_1$ and $\state_2$ the state after the termination of the first $\act{read-config}$ 
	of $\op_2$.
	%$\cvec{\pr_1}{\state_1}$ and $\cvec{\pr_2}{\state_2}$ at two states $\state_1$ and $\state_2$ 
	%in an execution $\EX$ of the algorithm  such that $\state_1$ appears before $\state_2$ in $\EX$. 
	Then the following properties hold: 
	%\begin{enumerate}
		%\item [$(a)$] 
		$(a)$ \textbf{Configuration Consistency}: 
		$\cvec{\pr_2}{\state_2}[i].cfg = \cvec{\pr_1}{\state_1}[i].cfg$,  for $ 1 \leq i \leq |\cvec{\pr_1}{\state_1}|$,
		%\item [$(b)$]
		$(b)$  
		\textbf{Seq. Prefix}: $\cvec{\pr_1}{\state_1}  \preceq_p \cvec{\pr_2}{\state_2}$, and
		%\item [$(c)$] 
		$(c)$  
		\textbf{Seq. Progress}: $\mu(\cvec{\pr_1}{\state_1}) \leq \mu(\cvec{\pr_2}{\state_2})$
		%; and 
		%\item [$(d)$]  
		%\nn{????$(d)$  $\cvec{\pr_2}{\state_2}[i]   = \cvec{\pr_1}{\state_1}[i]$,  for  $ 1 \leq i \leq \mu(\cvec{\pr_1}{\state_1})$????} 
	%\end{enumerate}
\end{theorem}

Given the properties satisfied by the reconfiguration algorithm of \ares{} 
and assuming that the DAP used satisfy Property~\ref{property:dap}, as presented
in Section \ref{ssec:dap}, then  we have the following result. 

\begin{theorem}[Atomicity]
	In  any execution $\EX$ of \ares{}, if in every configuration $c\in\gseq$,
	$\dagetdata{c}$, $\daputdata{c}{}$, and $\dagettag{c}$
	 %the DAP primitives  
	 satisfy Property~\ref{property:dap}, then ~\ares{} satisfy atomicity.
	%, given that the 
	%$\act{get-data}$, $\act{get-tag}$, and $\act{put-data}$ primitives used satisfy properties
	%\textbf{C1} and \textbf{C2} of Definition \ref{def:consistency}.
\end{theorem}

%In  \ares{},  each configuration 
%may implement the DAPs in a different way as stated below:

\begin{remark}
	Algorithm \ares{} satisfies atomicity even when the implementaton of the  DAPs in two 
	different configurations $c_1$ and $c_2$ are not the same, given that the $c_i.\act{get-tag}$,
	$c_i.\act{get-data}$, and the $c_i.\act{put-data}$ primitives 
	in each $c_i$ satisfy Property~\ref{property:dap}.  
\end{remark}

\subsection{Storage and Communication Costs for \ares{}.}\label{sec:safety:c}
We now briefly present the storage and communication costs associated with the presented DAPs.
%{\treas{}  %Due to space limitations the proofs appear in \cite{}.
Recall that by our assumption, the storage cost counts the size (in bits) of the coded elements 
%that are locally stored, which are 
stored in variable  $List$  at each server. We ignore the storage cost due to meta-data.
For  communication cost we measure the bits sent on the wire between the nodes.

\begin{theorem}\label{treas:performance}
 The \ares{} algorithm has: (i) storage cost $(\delta +1 )\frac{n}{k}$, (ii) communication 
cost for each write at most to $\frac{n}{k}$, and (iii) communication 
cost for each read at most $(\delta +2)\frac{n}{k}$.
\end{theorem}

\subsection{Latency Analysis for read and writes in \ares{}}\label{sec:safety:d}
Liveness properties cannot be specified for ~\ares{}, without restricting asynchrony  or the
rate of arrival of \act{reconfig} operations, or if the consensus protocol never terminates.
Here,  we provide some conditional performance analysis of the operation, based on 
%assumptions on the 
latency bounds on the message \nnrev{delay}{delivery}. %in the network.
 We assume that local computations take negligible time and the latency of an 
operation is  due to the delays in the messages exchanged during the execution. 
%Before proceeding with our 
%analysis we define an upper and lower communication bounds. 
We measure delays in \myemph{time units} of some global clock, which is visible only to an external viewer.
No process has access to the clock.
Let $\smdelay$ and $\lgdelay$ be the minimum and maximum durations taken by 
 messages, sent during an execution  of~\ares,  to reach their destinations.
%denote the minimum message delivery delay 
%between any two processes in the service; let $\lgdelay$ be the 
%maximum delivery delay. 
 Also, let $\opdelay{\op}$ denote the duration 
 of an operation (or action) $\op$. In the statements that follow, 
 we consider any execution $\EX$ of \ares, which contains $k$ \act{reconfig} operations.
 \kmk{For any configuration $c$ in an execution of~\ares{},  we assume that any 
 	$\consensus{c}.\act{propose}$ operation, takes at least $\opdelaymin{CN}$ time units.}
%  in it 
% and $\smdelay$ and $\lgdelay$ are the minimum and maximum durations taken by 
%any message sent during an execution to reach its destination.
% with the above time bounds 
%for messages transmitted. 
%%
%\myparagraph{Notations}
%Let $\smdelay$ and $\lgdelay$ be the minimum and maximum durations taken by 
%any message to go from one process to another.
%denote the minimum message delivery delay 
%between any two processes in the service; let $\lgdelay$ be the 
%maximum delivery delay. 
% Also, let $\opdelay{\op}$ denote the communication  delay of an operation (or action) $\op$. 
%The latency results are the following. 
%For the simplicity of our analysis, we assume that any 
%\act{propose} operation to a consensus instance takes at least   $\opdelaymin{CN}$ time units
 %and at most $\opdelaymax{CN}$. If $\op$ is any propose operation to a consensus service then 
%$\opdelaymin{CN} \leq \opdelay{\op} \leq \opdelaymax{CN}$.
 %time units.
\remove{
\begin{lemma}
\label{lem:opdelays}
Suppose $\pi$ and $\phi$ are operations of the type \act{put-config}, \act{read-next-config}, respectively, invoked by some non-faulty reconfiguration clients,  then the latency of these operations are bounded as follows: 
%\begin{itemize}
	%\item 
	$(i)$ $2\smdelay\leq \opdelay{\pi}\leq 2\lgdelay$
	%\item 
	and 
	$(ii)$ $2\smdelay\leq \opdelay{\phi}\leq 2\lgdelay$.
%\end{itemize}
\end{lemma}

From Lemma \ref{lem:opdelays} we can derive the delay of  a \act{read-config} action.
\begin{lemma}
	\label{lem:rcdelay}
	Let $\phi$ be a $\act{read-config}$ operation invoked by a non-faulty reconfiguration client, 
	with the input argument and returned values of $\phi$ as  $seq$ and  $seq'$, respectively. 
	If  $\mu = \mu(seq)$ and $\nu = \nu(seq')$, then 	
$4\smdelay(\nu-\mu+1)\leq \opdelay{\phi}\leq 4\lgdelay(\nu-\mu+1)$.
\end{lemma}

From Lemma \ref{lem:rcdelay} it is clear that the latency of a $\act{read-config}$ action 
is  dependent on the number of configurations installed since the last  finalized configuration known to the 
 recon client.
Let $\seqlen = \nu-\mu$ denote the number of newly installed configurations.
%Now let us examine when a new configuration gets inserted in the configuration 
%sequence by a \act{reconfig} operation.
Given the latency of a \act{read-config}, we can compute the minimum amount 
of time it takes for $k$ configurations to be installed.

\begin{lemma}
	\label{lem:configdelay}
	Let $\sigma$ be the last state of a fair execution of \ares{}, $\EX$. 
	Then $k$ configurations can be installed to $\cvec{}{\sigma}$, in time no less than
%	\begin{equation}
	$ \left(\opdelaymin{CN}+6\smdelay\right)k$.
%	\end{equation}
	%by the completion of its $\act{add-config}$ action 
%	in our execution construction.
\end{lemma}
}
\remove{
 In \ares{} a \act{reconfig} operation has 
four phases: $(i)$ $\act{read-config}(cseq)$,  reads the latest configuration sequence, 
$(ii)$ $\text{\act{add-config}}(cseq, c)$,  attempts to add  the new configuration 
at the end of the global sequence $\mathcal{G}_L$, 
$(iii)$ $\text{\act{update-config}}(cseq)$,   transfers the knowledge to the added configuration,
and 
$(iv)$  $\text{\act{finalize-config}}(cseq)$ finalizes the added configuration. 

%So, a new configuration is appended to the 
%end of the configuration sequence (and it becomes visible to any operation) during the 
%\act{add-config} action. 
 During the execution of \act{add-config} action, the recon client proposes to a consensus 
service  
to learn the configuration to accept, and then invokes a \act{put-config} action notify a quorum of servers in the configuration of the  decided configuration. 
%Any operation that is invoked after the \act{put-config} action 
%will observed the newly added configuration. 

When multiple reconfiguration operations  are invoked concurrently, each at a separate client, then it is possible 
that each of the clients  successfully append their  new  configurations to the end of $\mathcal{G}_L$. 
This is possible when 
the \act{read-config} action of each \act{reconfig} operation begins  after the completion of  \act{put-config}
action of another \act{reconfig} operation. 
}

%Next we 
The following lemma shows the maximum latency of a read or a write operation, invoked by any non-faulty client. 
From~\ares{} algorithm,  the latency of a read/write operation depends on the delays of the  DAPs  operations. 
%influence  the delay of a read and write operation. 
For our analysis we assume 
that all $\act{get-data}$, $\act{get-tag}$ and $\act{put-data}$ primitives use 
two phases of communication.  Each phase consists of a communication between the client and the servers.
%\kmkremove{Such an assumption is justified by atomic algorithms like ~\treas{} and ABD. }

\begin{lemma}
	\label{lem:dapdelays}
Suppose $\pi$,  $\phi$ and $\psi$ are operations of the type \act{put-data}, \act{get-tag} and  \act{get-data}, respectively, invoked by some non-faulty reconfiguration clients,  then the latency of these operations are bounded as follows: 
	$(i)$ $2\smdelay\leq \opdelay{\pi}\leq 2\lgdelay$; $(ii)$
	 $2\smdelay\leq \opdelay{\phi}\leq 2\lgdelay$; and $(iii)$
	  $2\smdelay\leq \opdelay{\psi}\leq 2\lgdelay$.
\end{lemma}
%Now we show the delay of a read or a write operation $\op$.

In the following lemma, we {estimate the time taken for a read or a write operation to complete,
	when it discovers $k$ configurations between its invocation and response steps.
%
%show that in a fair execution of ~\ares{} \nnrev{where there are at}{that contains} $k$ reconfiguration 
%operations, any read or write operation takes at most  $6\lgdelay\left(k+2\right)$.
 

\begin{lemma}
	\label{lem:rwdelay}
%Consider a execution of ~\ares{} where there are $k$ reconfiguration and a read or a write operation $\pi$ invoked by a non-faulty then   $6\lgdelay(k+1)$.
Consider any  execution of ~\ares{} where at most  $k$ reconfiguration operations are invoked.
Let $\sigma_s$ and $\sigma_e$ be the states before the invocation 
and after the completion step of a read/write operation $\op$,
in some fair execution $\EX$ of \ares{}. 
%If $k=\nu(\cvec{\pr}{\sigma_e}) - \mu(\cvec{\pr}{\sigma_s})$, 
Then we have 
$\opdelay{\op}\leq 6\lgdelay\left(k+2\right)$ to complete. 
\end{lemma}
\proofremove{
\begin{proof}
	Let $\state_s$ and $\state_e$ be the states before the invocation 
	and after the completion step of a read/write operation $\op$ by $\pr$ respectively,
	in some execution $\EX$ of \ares. 
	%Then $\op$ takes time at most:
%	\[
%		\opdelay{\op}\leq 6\lgdelay\left[\nu(\cvec{\pr}{\state_e}) - \mu(\cvec{\pr}{\state_s})+2\right]
%	\] 
	By algorithm examination we can 
	see that any read/write operation performs the following actions in this order:
	$(i)$  \act{read-config}, $(ii)$ \act{get-data} (or \act{get-tag}), $(iii)$ \act{put-data},
	and $(iv)$ \act{read-config}. Let $\state_1$ be the state when the first \act{read-config}, denoted by $\act{read-config}_1$, 
	action terminates. By Lemma \ref{lem:rcdelay} the action will take time:
	\[
		\opdelay{\act{read-config}_1} \leq 4\lgdelay(\nu(\cvec{\pr}{\state_1})-\mu(\cvec{\pr}{\state_s})+1)
	\]
	The $\act{get-data}$ action that follows the \act{read-config} (Lines Alg.~\ref{algo:writer}:\ref{line:rw:getdata:start}-\ref{line:rw:getdata:end}) is also executed at most $(\nu(\cvec{\pr}{\state_1})-\mu(\cvec{\pr}{\state_s})+1)$
	given that no new finalized configuration was discovered by the \act{read-config} action. 
	Finally, the \act{put-data}  and the second \act{read-config} actions of $\op$ may be invoked at most
	$(\nu(\cvec{\pr}{\state_e})-\nu(\cvec{\pr}{\state_1})+1)$ times, given that the \act{read-config} action discovers 
	one new configuration every time it runs. Merging all the outcomes, the total time of $\op$ can be at most:
	\begin{eqnarray*}
	\opdelay{\op} & \leq & 4\lgdelay(\nu(\cvec{\pr}{\state_1})-\mu(\cvec{\pr}{\state_s})+1) + 2\lgdelay(\nu(\cvec{\pr}{\state_1})-\mu(\cvec{\pr}{\state_s})+1) + (4\lgdelay+2\lgdelay)(\nu(\cvec{\pr}{\state_e})-\nu(\cvec{\pr}{\state_1})+1) \\
%	 & \leq & 6\lgdelay\nu(\cvec{\pr}{\state_1})-6\lgdelay\mu(\cvec{\pr}{\state_s})+6\lgdelay\nu(\cvec{\pr}{\state_e})-6\lgdelay\nu(\cvec{\pr}{\state_1})+12\lgdelay \\
	 & \leq & 6\lgdelay\left[\nu(\cvec{\pr}{\state_e}) - \mu(\cvec{\pr}{\state_s})+2\right] \leq 6D(k+1)
	\end{eqnarray*}
where  $\nu(\cvec{\pr}{\state_e}) - \mu(\cvec{\pr}{\state_s})\leq k + 1$ since there there can be at most $k$ new configurations installed. and the result of the lemma follows.
\end{proof}
}

It remains now to examine the conditions under which a read/write operation may “catch up” with an infinite number of reconfiguration operations.
For the sake of a worst case analysis we will assume that reconfiguration operations suffer 
the minimum delay $d$, whereas read and write operations suffer the maximum
delay $D$ in each message exchange. Also, we assume that any consensus operation takes the least amount of time to complete $\opdelaymin{CN}$.
The following theorem is the main result of this section, in which we define the relation between $\opdelaymin{CN}$, $d$ and $D$
%show that in any fair execution of ~\ares{}, if the  consensus operations are ``slow enough'' then  
so to guarantee that any read or write issued by a non-faulty client always terminates.

\begin{theorem}
%Consider a  read or  write operation $\pi$, invoked by a non-faulty client, in a fair and well-formed execution of ~\ares{}. Suppose at the point of invocation of $\pi$ the client has $|cseq| = p$. Then if $\opdelaymin{CN} \geq 6D(p + 2) -5d$ the operation $\pi$ completes.
 Suppose  $\opdelaymin{CN} \geq 3(6D-\smdelay)$, then  any  read or write operation $\op$ completes in any execution  $\EX$ of 
\ares{}  for any number of reconfiguration operations in $\EX$.
%$\smdelay \geq \frac{3\lgdelay}{k}-\frac{\opdelay{CN}}{2(k+2)}$
\end{theorem}

\remove{

It remains now to examine if a read/write operation may ``catch up'' with any ongoing 
reconfigurations. 
For the sake of a worst case analysis we will assume that reconfiguration operations
may communicate respecting the minimum delay $\smdelay$, whereas read and write 
operations suffer the maximum delay $\lgdelay$ in each message exchange. We will
split our analysis into three directions, with respect to the number of configurations 
installed $k$, and the bound on the minimum delay $\smdelay$:
$(i)$  $k$ is finite, and $\smdelay$ may be unbounded small; $(ii)$ 
$k$ is infinite, and $\smdelay$ may be unbounded small;
$(iii)$  $k$ is infinite, and $\smdelay$ can be bounded.

\myparagraph{$k$ is finite, and $\smdelay$ may be unbounded small.}
In this case we assume a finite number of installed configurations. Also,
as the $\smdelay$ is unbounded, it follows that reconfigurations may be 
installed almost instantaneously. Let us first examine what is the maximum 
delay bound of a any read/write operation. 

\myparagraph{$k$ is infinite, and $\smdelay$ is bounded.}
We will compute the bound on $\smdelay$ with respect to the $\lgdelay$ 
and the number of configurations to be installed $k$ if we want to allow a 
read/write operation to reach ongoing reconfigurations. 

\begin{lemma} \label{lem:delaybound}
	A read/write operation $\op$ may terminate in any execution $\EX$ of 
	\ares{} given that $k$ configurations are installed during $\op$, if
	$
	\smdelay \geq \frac{3\lgdelay}{k}-\frac{\opdelaymin{CN}}{2(k+2)}
	$
\end{lemma}
%\proofremove{
\begin{proof}
	By Lemma \ref{lem:configdelay}, $k$ configurations may be installed in at least:
		$\opdelay{k} \geq 4\smdelay\sum_{i=1}^{k}i+ k\left(\opdelaymin{CN}+2\smdelay\right)$.	
	Also by Lemma \ref{lem:rwdelay}, we know that operation $\op$ takes at most 
$	\opdelay{\op}\leq 6\lgdelay\left(\nu(\cvec{\pr}{\state_e}) - \mu(\cvec{\pr}{\state_s})+2\right)$.
	Assuming that $k=\nu(\cvec{\pr}{\state_e}) - \mu(\cvec{\pr}{\state_s})$, the total number of 
	configurations observed during $\op$, then $\op$ may terminate before a $k+1$ configuration 
	is added in the configuration sequence if 
	  $6\lgdelay(k+2) \leq  4\smdelay\sum_{i=1}^{k}i+ k\left(\opdelaymin{CN}+2\smdelay\right)$ then we have
	  $d \geq \frac{3\lgdelay}{k}-\frac{\opdelaymin{CN}}{2(k+2)}$.
	And that completes the lemma. 
\end{proof}
%}
}

%\begin{theorem*}{\bf \ref{safety:ares:treas}} (Atomicity)
%	Algorithm \ares-\treas{} implements a reconfigurable atomic storage service, given that the 
%	$\act{get-data}$, $\act{get-tag}$, and $\act{put-data}$ primitives used satisfy properties
%	\textbf{C1} and \textbf{C2} of Definition \ref{def:consistency}.
%\end{theorem*}
%
%\begin{proof}
%The figure above \ref{fig:reconfig:ares:treas}
%\end{proof}
