
%
% 
% This abstraction enables us to prove the safety and liveness properties of such algorithms based on the properties of these primitives. 
% This abstraction servers us a two-fold 
% purpose: $(i)$ by expressing several atomicity emulation algorithm in terms of the primitives allows us to prove safety and liveness based on their properties $(iii)$ shows how such algorithms can be adopted to our ARES algorithm and prove their safety and liveness without; and $(iii)$ exposes the intuition that the underlying atomicity algorithm can  be different from configuration to configuration.
% For version control of the  object values  we use tags.  
% 
 
 
 %Let $<_\tau$ and $\leq_\tau$ be the appropriate comparison relationships used by any algorithm 
 %that utilizes logical timestamps. Then 
 %atomicity properties can be expressed in terms of the tags written and returned by write and read 
 %operations respectively. 
 %For a write operation $\wrt$ we denote by $\tg{\wrt}$ the tag that is 
 %used by $\wrt$ and for a read $\rd$ we denote by $\tg{\rd}$ the tag that is returned by $\rd$
 %\footnote{Note that the values written or returned by write of read operations can be mapped easily  
 %to the tags they write or return.}.	The partial ordering among the  operations  can then be induced from the partial ordering among the tags. 
 %using  tags in the following way: (i) for any two write 
 %operations $\wrt_1$, $\wrt_2$, if  $\wrt_1\prec\wrt_2$, then $\tg{\wrt_1}<_\tau\tg{\wrt_2}$,
 %(ii) For any operation $\op_1$,  and any read operation $\rd_2$, if $\op_1\prec\rd_2$, then
 %$\tg{\op_1}\leq_\tau\tg{\rd_2}$.

\proofremove{
 \begin{proof}
 We  prove the atomicity by proving properties $P1$, $P2$ and $P3$ appearing in Lemma \ref{XXX} for any execution of the algorithm.
					
	\emph{Property $P1$}: Consider two operations $\phi$ and $\pi$ such that $\phi$ completes before $\pi$ is invoked. 
	We need to show that it cannot be  the case that $\pi \prec \phi$. We break our analysis into the following four cases:

	Case $(a)$: {\em Both $\phi$ and $\pi$ are writes}. The $\daputdata{c}{*}$ of $\phi$ completes before 
	$\pi$ is invoked. 
	%which implies that by well-formedness 
	By property $C1$ the tag $\tg{\pi}$ returned by the $\dagetdata{c}$ at $\pi$ is 
	at least as large as $\tg{\phi}$. Now, 
	%since $\tg{\pi}$ is larger than $t_{\phi}$, by the steps of 
	since $\tg{\pi}$ is incremented by the write operation then $\pi$ puts a tag $\tg{\pi}'$ such that
	$\tg{\phi} < \tg{\pi}'$ and hence we cannot have $\pi \prec \phi$.
	
	Case $(b)$: {\em $\phi$ is a write and  $\pi$ is a read}. In execution $\EX$ since 
$\daputdata{c} {\tup{t_{\phi}, *}}$ of $\phi$ completes 
	before the $\dagetdata{c}$ of $\pi$ is invoked, by 
	%the well-formedness 
	property $C1$ the tag $\tg{\pi}$ obtained from the above
	$\dagetdata{c}$ is at least as large as $\tg{\phi}$. Now $\tg{\phi} \leq \tg{\pi}$ implies that we cannot have $\pi \prec \phi$.
	
	Case $(c)$: {\em $\phi$ is a read and  $\pi$ is a write}.  Let the id of the writer that invokes $\pi$ we $w_{\pi}$.  
	The 
$\daputdata{c}{\tup{\tg{\phi}, *}}$  call of $\phi$ completes
	before  $\dagettag{c}$ of $\pi$ is initiated. Therefore, by 
	%the well-formedness 
	property $C1$ %of data-primitives the above 
	$\act{get-tag}(c)$ returns $\tg{}$ such that, $\tg{\phi} \leq \tg{}$. Since $\tg{\pi}$ is equal to $(\tg{}.z + 1, w_{\pi})$ 
	by design of the algorithm, hence $\tg{\pi} > \tg{\phi}$ and we cannot have $\pi \prec \phi$.
	
	Case $(d)$: {\em Both $\phi$ and $\pi$ are reads}. In execution $\EX$  
the $\daputdata{c}{\tup{t_{\phi}, *}}$ is executed as a part of $\phi$ and 
	completes before $\dagetdata{c}$ is called in $\pi$. By 
	%the well-formedness
	 property $C1$ of the data-primitives, 
	we have $\tg{\phi} \leq \tg{\pi}$ and hence we cannot have $\pi \prec \phi$.
	
	\emph{Property $P2$}: Note that because $\tsSet$ is well-ordered we can show that this property by first showing that
	every write has a unique tag. This means any two pair of writes can be ordered. Now, a read can be ordered . Note that 
	a read can be ordered w.r.t. to any write operation trivially if the respective tags are different, and by definition, if the 
	tags are equal the write is ordered before the read.
	
	Now observe that two tags generated from two write operations from different writers are necessarily distinct because of the 
	id component of the tag. Now if the operations, say $\phi$ and $\pi$ are writes  from the same writer then by 
	well-formedness property the second operation is invoked after the first completes, say without loss of generality $\phi$ completes before 
	$\pi$ is invoked.   In that case the integer part of the tag of $\pi$ is higher 
	%because the well-formedness 
	by property  $C1$, and since the $\dagettag{c}$  is followed by $\daputdata{c}{*}$. Hence $\pi$ is ordered after $\phi$. 
	
	\emph{Property $P3$}:  This is clear because the tag of a reader is defined by the tag of the value it returns by property (b).
	Therefore, the reader's immediate previous value it returns. On the other hand if  does 
	note return any write operation's value it must return $v_0$.
 \end{proof}
}


