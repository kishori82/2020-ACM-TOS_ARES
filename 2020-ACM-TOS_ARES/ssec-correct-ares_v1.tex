% The correctness of \ares{} highly depends on the way the configuration 
% sequence is constructed at each client process. Let $\cvec{p}{\state}$ 
% denote the configuration sequence $cseq$ at process $p$ in a state $\state$ and 
% $\mu(\cvec{\pr}{\state})$ the index of the last finalized configuration in $\cvec{p}{\state}$. 
% Then the following properties are preserved by the reconfiguration service:

% \begin{theorem}
% 	Let $\op_1$ and $\op_2$ two 
% 	%read/write/install operations 
% 	completed \act{read-config} actions invoked by processes $\pr_1, \pr_2\in\idSet$ 
% 	respectively, such that $\op_1\bef\op_2$ in an execution $\EX$. 
% 	Let $\state_1$ the state after the response 
% 	step of $\op_1$ and $\state_2$ the state after the response step 
% 	%after the completion of $\op_1$ and $\state_2$ the state after the termination of the first $\act{read-config}$ 
% 	of $\op_2$.
% 	%$\cvec{\pr_1}{\state_1}$ and $\cvec{\pr_2}{\state_2}$ at two states $\state_1$ and $\state_2$ 
% 	%in an execution $\EX$ of the algorithm  such that $\state_1$ appears before $\state_2$ in $\EX$. 
% 	Then the following properties hold: 
% 	%\begin{enumerate}
% 		%\item [$(a)$] 
% 		$(a)$ \textbf{Configuration Consistency}: 
% 		$\cvec{\pr_2}{\state_2}[i].cfg = \cvec{\pr_1}{\state_1}[i].cfg$,  for $ 1 \leq i \leq |\cvec{\pr_1}{\state_1}|$,
% 		%\item [$(b)$]
% 		$(b)$  
% 		\textbf{Seq. Prefix}: $\cvec{\pr_1}{\state_1}  \preceq_p \cvec{\pr_2}{\state_2}$, and
% 		%\item [$(c)$] 
% 		$(c)$  
% 		\textbf{Seq. Progress}: $\mu(\cvec{\pr_1}{\state_1}) \leq \mu(\cvec{\pr_2}{\state_2})$
% 		%; and 
% 		%\item [$(d)$]  
% 		%\nn{????$(d)$  $\cvec{\pr_2}{\state_2}[i]   = \cvec{\pr_1}{\state_1}[i]$,  for  $ 1 \leq i \leq \mu(\cvec{\pr_1}{\state_1})$????} 
% 	%\end{enumerate}
% \end{theorem}

Given the properties satisfied by the reconfiguration algorithm of \ares{} 
% and assuming that the DAP used satisfy Property~\ref{property:dap}, as presented
% in Section \ref{ssec:dap}, then  we have the following result. 
in any execution, we can now proceed to examine whether our algorithm satisfies the safety (atomicity) conditions. The
propagation of the information of the distributed object is achieved using the get-tag, get-data, and
put-data actions. We assume that the DAP used satisfy  Property~\ref{property:dap} as presented in
Section \ref{ssec:dap}, and we will show that, given such assumption, \ares{} satisfies atomicity.

We begin with a lemma stating that if a reconfiguration operation retrieves a configuration
sequence of length $k$ during its $\act{read-config}$ action, then it installs/finalizes the $k + 1$ configuration in the global configuration sequence.

\begin{lemma}%{\bf 
	\label{lem:recon:incremental}
	Let $\op$ be a complete reconfiguration operation by a reconfigurer $\rec$ in an execution $\EX$ of \ares{}.
	if $\state_1$ is the state in $\EX$ following the termination of the $\act{read-config}$ action during $\op$,
	then $\op$ invokes a $\act{finalize-config}(\cvec{\rec}{\state_2})$ at a state $\state_2$ in $\EX$, 
	with $\nu(\cvec{\rec}{\state_2}) = \nu(\cvec{\rec}{\state_1}) + 1$.
	%		
	%Let $\rho$ denote a complete reconfiguration operation in an execution $\EX$ of \ares{} and let $\mathbf{x}$ be the 
	%configuration sequence returned by  the $\act{read-config}$ step, $a_1$  in $\rho$. Then if $\mathbf{y}$ is a reconfiguration 
	%sequence returned by another $\act{read-config}$ step $a_2$ that appears in $\EX$ after $a_1$ completes then during $\rho$ the call $\act{finalize-config}( \mathbf{y}( \nu(\mathbf{x}) + 1)$ is executed.
\end{lemma}

\begin{proof}
	This lemma follows directly from the implementation of the $\act{reconfig}$ operation. 
	Let $\op$ be a reconfiguration operation $\act{reconfig}(c)$. At first, $\op$ invokes a
	$\act{read-config}$ to retrieve a latest value of the global configuration sequence, $\cvec{\rec}{\state_1}$, 
	in the state $\state_1$ in $\EX$. During the $\act{add-config}$ action, $\op$ proposes the addition of $c$, 
	and appends at the end of $\cvec{\rec}{\state_1}$ the decision $d$ of the consensus protocol. 
	Therefore, if $\cvec{\rec}{\state_1}$ is extended by $\tup{d, P}$ (Line A~\ref{algo:reconfigurer}:\ref{line:addconfig:assign}), and hence the 
	$\act{add-config}$ action returns a configuration sequence $\cvec{\rec}{\state_1'}$ with length 
	$\nu(\cvec{\rec}{\state_1'})= \nu(\cvec{\rec}{\state_1}) + 1$. As $\nu(\cvec{\rec}{\state_1'}$ does not change 
	during the $\act{update-config}$ action, then $\cvec{\rec}{\state_1'}$ is passed to the $\act{finalize-config}$ action at state $\state_2$,
	and hence $\cvec{\rec}{\state_2}=\cvec{\rec}{\state_1'}$. Thus, $\nu(\cvec{\rec}{\state_2})=\nu(\cvec{\rec}{\state_1'})= \nu(\cvec{\rec}{\state_1}) + 1$ and the lemma follows.
\end{proof}

The next lemma states that only some reconfiguration operation $\op$ may finalize a configuration $c$ at index $j$ in a configuration sequence $\pr.cseq$ at any process $\pr$.
%, is finalized by some . 
To finalize $c$, the lemma shows that $\op$ must witness a configuration
sequence such that its last finalized configuration appears at an index $i<j$ in the configuration sequence $\pr.cseq$. In other words,  reconfigurations always finalize configurations that are ahead from their
latest observed final configuration, and it seems like ``jumping'' from one final configuration to the next.

\begin{lemma} %{\bf 
\label{lem:recon:jump}
Suppose $\EX$ is an execution of \ares{}. 
%Let $\cvec{*}{\state}$ be a configuration sequence returned by any $\act{read-config}$ action at state $\state$ in $\EX$,
 %and let $\cvec{*}{\state}[j].status = F$, for some $0 < j \leq \nu( \cvec{*}{\state})$. 
 For any state $\state$ in $\EX$, if $\status{\cvec{\pr}{\state}[j]} = F$ for some process $\pr\in\idSet$, 
 then there exists a $\act{reconfig}$ operation $\op$ by a reconfigurer $\rec\in\recSet$, such that
 (i) $\rec$ invokes $\act{finalize-config}(\cvec{\rec}{\state'})$ during $\op$ at some state $\state'$ in $\EX$, 
 (ii) $\nu(\cvec{\rec}{\state'}) = j$, and (iii) $\mu(\cvec{\rec}{\state'}) < j$. 
% 
% there exists a $\act{recon}$ operation $\rho$ that introduces $\cvec{}{}[j]$ 
% such that there exists  a configuration in $\cvec{rc}{}$ returned by a $\act{read-config}$ of $\rho$,  such that, $\cvec{rc}{}[k]$  $ k < j$ and  $\cvec{rc}{}[k].status = F$.
\end{lemma}
%
%\begin{lemma} \label{lem:monotonicity:cseq}
%If $cseq_1$ and $cseq_2$ are the values returned by two executions of  $\act{read-config}(c)$, where $c \in \mathcal{C}$ in an execution $\EX$ of \ares{}; and $cseq_1$ is returned  before the procedure call that returned $cseq_2$ is invoked in $\EX$  then $cseq_1 \preceq_p cseq_2$.  \blue{use this template}
%\end{lemma}

\begin{proof}
	A process sets the status of a configuration $c$ to $F$ in two cases: (i) either during 
	a $\act{finalize-config}(seq)$ action such that $\nu(seq) = \tup{c,P}$ (Line A\ref{algo:reconfigurer}:\ref{line:status:finalize}), or 
	(ii) when it receives $\tup{c,F}$ from a server $s$ during a $\act{read-next-config}$ action. 
	Server $s$ sets the status of a configuration $c$ to $F$ only if it receives a message that contains $\tup{c,F}$
	(Line A\ref{algo:server}:\ref{line:server:finalize}). So, (ii) is possible only if $c$ is finalized during a $\act{reconfig}$ operation. 
	
	Let, w.l.o.g., $\op$ be the first reconfiguration operation that finalizes $\config{\cvec{\pr}{\state}[j]}$. 
	To do so, process $\rec$
	invokes $\act{finalize-config}(\cvec{\rec}{\state_1'})$ during $\op$, at some state $\state'$ that appears
	before $\state$ in $\EX$. By Lemma \ref{lem:unique}, $\config{\cvec{\pr}{\state}[j]} = \config{\cvec{\rec}{\state'}[j]}$.
	Since, $\rec$ finalizes $\cvec{\rec}{\state'}[j]$, then  this is the last entry of $\cvec{\rec}{\state'}$ and 
	hence $\nu(\cvec{\rec}{\state'}) = j$. 
	Also, by Lemma \ref{lem:recon:jump} it follows that the 
	$\act{read-config}$ action of $\op$ returned a configuration $\cvec{\rec}{\state''}$ in some state 
	$\state''$ that appeared before $\state'$ in $\EX$, such that $\nu(\cvec{\rec}{\state''}) < \nu(\cvec{\rec}{\state'})$.
	Since by definition, $\mu(\cvec{\rec}{\state''})\leq\nu(\cvec{\rec}{\state''})$, then $\mu(\cvec{\rec}{\state''}) < j$.
	However, since only $\tup{c, P}$ is added to $\cvec{\rec}{\state''}$ to result in $\cvec{\rec}{\state'}$,
	then $\mu(\cvec{\rec}{\state''}) = \mu(\cvec{\rec}{\state'})$. Therefore, $\mu(\cvec{\rec}{\state'}) < j$ as well 
	and the lemma follows.
\end{proof}


We now reach an important lemma of this section. By \ares{}, before a read/write/reconfig operation completes 
it propagates the maximum tag it discovered by executing the $\act{put-data}$ action in the last 
configuration of its local configuration sequence (Lines A\ref{algo:reconfigurer}:\ref{line:addconfig:put}, A\ref{algo:writer}:\ref{line:writer:prop}, A\ref{algo:writer}:\ref{line:reader:prop}). 
When a subsequent operation is invoked, it reads the latest configuration 
sequence by beginning from the last finalized configuration in its local sequence and invoking $\act{read-data}$ to
all the configurations until the end of that sequence. The lemma shows that the latter operation 
will retrieve a tag which is higher than the tag used in the $\act{put-data}$ action of any preceding operation.

% For the following proof we use the notation $\nu^c(\cvec{\pr}{\st}) = \config{\cvec{\pr}{\st}[\nu(\cvec{\pr}{\st})]}$. In other words, $\nu^c(\cvec{\pr}{\st})$ denotes 
% the last configuration in the sequence $\cvec{\pr}{\st}$.

\begin{lemma}%{\bf  
	\label{lem6}
	Let $\pi_1$ and $\pi_2$ be two completed read/write/reconfig  operations invoked by processes
	$p_1$ and $p_2$ in $\idSet$,  in an execution  $\EX$ of \ares{}, such that, $\pi_1\bef\pi_2$. 
	If $c_1.\act{put-data}(\tup{\tg{\op_1}, v_{\op_1}})$ is the last \act{put-data} action of $\pi_1$
	and $\state_2$ is the state in $\EX$ after the completion of the first $\act{read-config}$ action of $\pi_2$, 
	then there exists a $c_2.\act{put-data}(\tup{\tg{},v})$ action in some configuration 
	$c_2 = \config{\cvec{\pr_2}{\state_2}[k]}$, for $\mu(\cvec{\pr_2}{\state_2}) \leq k \leq
	\nu(\cvec{\pr_2}{\state_2})$, such that (i) it completes in a state $\state'$ before $\state_2$ in $\EX$, and (ii) $\tg{}\geq\tg{\op_1}$.
\end{lemma}

\begin{proof}
Note that from the definitions of $\nu(\cdot)$ and $\mu(\cdot)$, we have  
$\mu(\cvec{\pr_2}{\state_2}) \leq \nu(\cvec{\pr_2}{\state_2})$.  Let $\state_1$ be the state in $\EX$ 
after the completion of $c_1.\act{put-data}(\tup{\tg{\op_1}, v_{\op_1}})$ and 
$\state'_1$ be the state in $\EX$ 
%immediately 
following the response step of $\pi_1$. Since any operation executes $\act{put-data}$ 
on the last discovered configuration then $c_1$ is the last configuration found in 
$\cvec{\pr_1}{\state_1}$, and hence $c_1 =  \config{\cvec{\pr_1}{\st_1}[\nu(\cvec{\pr_1}{\st_1})]}$. %\nu^c(\cvec{\pr_1}{\state_1})$.  
By Lemma~\ref{lem:final:monotonic}   we have
$\mu(\cvec{\pr_1}{\state_1}) \leq \mu(\cvec{\pr_1}{{\state}'_1})$ 
and by Lemma~\ref{lem:finalconf} we have 
$\mu(\cvec{\pr_1}{{\state}'_1}) \leq \mu(\cvec{\pr_2}{{\state}_2})$, since $\pi_2$ (and thus its first \act{read-config}
action) is invoked after ${\state}'_1$ (and thus after the last \act{read-config} action during $\op_1$). %, i.e., after $\pi_1$ completes  and 
Hence, combining the two implies that
$\mu(\cvec{\pr_1}{\state_1}) \leq \mu(\cvec{\pr_2}{{\state}_2})$.   Now from the last implication and the first statement we have $\mu(\cvec{\pr_1}{{\state}_1}) \leq \nu(\cvec{\pr_2}{{\state}_2})$.
Therefore,  it remains to examine whether the last finalized configuration witnessed by $\pr_2$ appears before or after $c_1$, i.e.:
$(a)$ $\mu(\cvec{\pr_2}{\state_2}) \leq \nu(\cvec{\pr_1}{{\state}_1})$ and 
$(b)$ $\mu(\cvec{\pr_2}{\state_2}) > \nu(\cvec{\pr_1}{{\state}_1})$.
% Let for appreviation denote by $\mu_i = \mu(\cvec{\pr_i}{\state_i})$, 
% and $\nu_i = \nu(\cvec{\pr_i}{\state_i})$.

\vspace{1em}

\noindent{\textbf{Case $(a)$:}} Since $\op_1\bef\op_2$
%operation $\pi_2$ is invoked after $\pi_1$ completes 
then, by Theorem \ref{thm:recon:properties}, $\cvec{\pr_2}{\state_2}$ value returned
by $\act{read-config}$ at $p_2$ during the execution of $\pi_2$ satisfies $\cvec{\pr_1}{\state_1}  \preceq_p \cvec{\pr_2}{\state_2}$. 
%
Therefore, $\nu(\cvec{\pr_1}{{\state}_1})  \leq \nu(\cvec{\pr_2}{{\state}_2})$, and hence in this case 
$\mu(\cvec{\pr_2}{\state_2}) \leq \nu(\cvec{\pr_1}{{\state}_1})  \leq \nu(\cvec{\pr_2}{{\state}_2})$. 
Since $c_1$ is the last configuration in $\cvec{\pr_{1}}{\state_1}$, then it has index $\nu(\cvec{\pr_1}{{\state}_1})$.
So if we take $c_2 = c_1$ then the $c_1.\act{put-data}(\tup{\tg{\op_1}, v_{\op_1}})$ action trivially 
satisfies both conditions of the lemma as: (i) it completes in state $\state_1$ which appears before $\state_2$, and
(ii) it puts a pair $\tup{\tg{}, v}$ such that $\tg{}=\tg{\op_1}$. 
%if $\pi_2$ is either a  write or a read operation then   $\act{get-tag}$  or $\act{get-data}$   is called on each configuration 
%  $\cvec{p_2}{\sigma_2}[i]$ for $ \mu(\cvec{p_2}{\sigma_2}) \leq i \leq \nu(\cvec{p_2}{\sigma_2})$ .  Note that 
%  $\mu(\cvec{\pr_2}{\state_2}) \leq \nu(\cvec{\pr_1}{{\state}_1})  \leq \nu(\cvec{\pr_2}{{\state}_2}) $ 
%  then by Definition~\ref{def:consistency} and because $\act{put-data}$ corresponding to  $\sigma_1$ at $p_1$ completes before $\act{get-tag}$ or $\act{get-data}$ the claim of the lemma holds.  
  \vspace{1em}
  
\noindent{\textbf{Case $(b)$:}} This case is possible if there exists a reconfiguration client $\rec$ that invokes $\act{reconfig}$ operation $\rho$, 
%say corresponding to some reconfiguration operation $\rho$,  that 
during which it executes the 
$\act{finalize-config}( \cvec{\rec}{*})$ that finalized configuration with index 
$\nu(\cvec{\rec}{*}) = \mu(\cvec{\pr_2}{\state_2})$. 
Let $\state$ be the state immediately after the $\act{read-config}$ of $\rho$. Now, we consider two sub-cases: $(i)$ $\state$ appears before $\state_1$ in $\EX$, or  $(ii)$ $\state$ appears after $\state_1$ in $\EX$.
  \vspace{1em}
  
 \noindent{\emph{Subcase $(b)(i)$:}} Since $\act{read-config}$ at $\state$ completes before the invocation of last $\act{read-config}$ of operation  $\pi_1$ then, either
   $\cvec{\rec}{\state}  \prec_p \cvec{\pr_1}{\state_1}$, or $\cvec{\rec}{\state}  = \cvec{\pr_1}{\state_1}$ due to Lemma~\ref{lem:prefix}.  
  Suppose  $\cvec{\rec}{\state}  \prec_p \cvec{\pr_1}{\state_1}$,  then according to Lemma~\ref{lem:recon:incremental} $\rec$  executes $\act{finalize-config}$  on configuration sequence  $\cvec{\rec}{*}$ with  $\nu(\cvec{\rec}{*}) = \nu(\cvec{\rec}{\state}) + 1$. 
  %configuration as argument  but since
 %$\rho_0$ installs $\mu^c(\cvec{p_2}{\sigma_2})$ then 
  Since $\nu(\cvec{\rec}{*}) = \mu(\cvec{\pr_2}{\state_2})$, then
  $\mu(\cvec{p_2}{\sigma_2}) = \nu(\cvec{rc}{\sigma}) + 1$. 
  If however, $\cvec{\rec}{\state}  \prec_p \cvec{\pr_1}{\state_1}$, then 
  $\nu(\cvec{\rec}{\state}) < \nu(\cvec{\pr_1}{\state_1})$ and thus 
  $\nu(\cvec{\rec}{\state})+1 \leq \nu(\cvec{\pr_1}{\state_1})$. 
  This implies that
%  But this implies 
 %  $\nu(\cvec{rc}{\sigma})  + 1 \leq  \nu(\cvec{p_1}{\sigma_1}) $ which
      $\mu(\cvec{p_2}{\sigma_2})  \leq  \nu(\cvec{p_1}{\sigma_1})$  which 
      contradicts our initial assumption for this case that $\mu(\cvec{p_2}{\sigma_2}) >  \nu(\cvec{p_1}{\sigma_1})$. So this sub-case is impossible.
      %a contradiction. 

  Now suppose,  that $\cvec{\rec}{\state}  = \cvec{\pr_1}{\state_1}$. Then it follows that 
  $\nu(\cvec{\rec}{\state})  = \nu(\cvec{\pr_1}{\state_1})$, and that  
  $\mu(\cvec{p_2}{\sigma_2}) = \nu(\cvec{p_1}{\sigma_1}) + 1$ in this case.
  Since $\state_1$ is the state after the last $\act{put-data}$ during $\op_1$, then if $\state_1'$ is the state after the completion of the last $\act{read-config}$ of $\op_1$ (which follows the \act{put-data}), it must be the case that $\cvec{\pr_1}{\state_1} = \cvec{\pr_1}{\state_1'}$. So, during its last $\act{read-config}$ process  $p_1$ does not read the configuration indexed at $ \nu(\cvec{p_1}{\sigma_1}) + 1$. This means that the $\act{put-config}$ completes in $\rho$ at state ${\state}_{\rho}$ after ${\state}'_1$ and the 
  $\act{update-config}$ operation is invoked at state $\state_{\rho}'$ after $\state_{\rho}$ with 
  a configuration sequence $\cvec{\rec}{\state_{\rho}'}$.
  During the update operation $\rho$ invokes $\act{get-data}$ operation in every configuration 
  $\config{\cvec{\rec}{\state_{\rho}'}[i]}$, for $\mu(\cvec{\rec}{\state_{\rho}'})\leq i\leq\nu(\cvec{\rec}{\state_{\rho}'})$.
  Notice that $\nu(\cvec{\rec}{\state_{\rho}'})= \mu(\cvec{p_2}{\sigma_2}) = \nu(\cvec{p_1}{\sigma_1}) + 1$
  and moreover the last configuration of $\cvec{\rec}{\state_{\rho}'}$ was just added by $\rho$ and it is 
  not finalized. From this it follows that $\mu(\cvec{\rec}{\state_{\rho}'})<\nu(\cvec{\rec}{\state_{\rho}'})$,
  and hence $\mu(\cvec{\rec}{\state_{\rho}'})\leq\nu(\cvec{p_1}{\sigma_1}) $.
  Therefore,  $\rho$ executes $\act{get-data}$ in configuration $\config{\cvec{\rec}{\state_{\rho}'}[j]}$ for 
  $j = \nu(\cvec{p_1}{\state_1}) $. Since $\pr_1$ invoked $\act{put-data}(\tup{\tg{\op_1},v_{\op_1}})$ 
  at the same configuration $c_1$, 
  %$\nu^c(\cvec{p_1}{\state_1})$, 
  and completed in a state $\state_1$ before $\state_{\rho}'$, then by
  \textbf{C1} of Property~\ref{property:dap}, 
  it follows that the $\act{get-data}$ action will 
  return a tag $\tg{}\geq \tg{\op_1}$. Therefore, the maximum tag that $\rho$ discovers is 
  $\tg{max}\geq\tg{}\geq\tg{\op_1}$. Before invoking the $\act{finalize-config}$ action, $\rho$ 
  invokes %$\nu^c(\cvec{\rec}{\state_{\rho}'}).\act{put-data}(\tup{\tg{max}, v_{max})}$
  $c_1.\act{put-data}(\tup{\tg{max}, v_{max})}$. Since 
  $\nu(\cvec{\rec}{\state_{\rho}'})= \mu(\cvec{p_2}{\state_2})$, and since by Lemma \ref{lem:unique},
  then the action $\act{put-data}$ is invoked in a configuration $c_2 = \config{\cvec{p_2}{\sigma_2}[j]}$ 
  such that $j=\mu(\cvec{p_2}{\sigma_2})$. Since the $\act{read-config}$ action of $\op_2$ 
  observed configuration $\mu(\cvec{p_2}{\sigma_2})$, then it must be the case that $\state_2$ 
  appears after the state where the $\act{finalize-config}$ was invoked and therefore after the 
  state of the completion of the $\act{put-data}$ action during $\rho$. Thus, in this case both 
  properties are satisfied and the lemma follows. 
%  in state ${\state}''_1$ after ${\state}'_1$ and hence after $\state_1$. 
%  
%  
%  So by Definition~\ref{def:consistency} it returns a tag $t > tag(\pi_1)$. 
  \vspace{1em}

\noindent{\emph{Subcase $(b)(ii)$:}} Suppose 
%$\act{read-config}$ 
in this case that $\state$ occurs in $\EX$ after $\sigma_1$. In this case the 
   last $\act{put-data}$ in $\pi_1$ completes   before the invocation of the  $\act{read-config}$ in $\rho$ in execution $\EX$. 
   Now we can argue recursively, $\rho$ taking the place of operation $\pi_2$,  that $\mu(\cvec{\rec}{{\state}}) \leq \nu(\cvec{\rec}{{\state}})$ and 
therefore,  we consider  two cases:
$(a)$ $\mu(\cvec{\rec}{\state}) \leq \nu(\cvec{\pr_1}{{\state}_1})$ and 
$(b)$ $\mu(\cvec{\rec}{\state}) > \nu(\cvec{\pr_1}{{\state}_1})$. 
Note that there are finite number of operations  invoked in $\EX$  before $\pi_2$ is invoked, and hence  the statement of the lemma can be shown to hold by a sequence of inequalities.
   %
\end{proof}

The following lemma shows the consistency of operations as long as the DAP used satisfy 
Property~\ref{property:dap}. %properties \textbf{C1} and \textbf{C2}.

\begin{lemma} %{\bf \ref{dap:C1:C2}} 
	 Let $\op_1$ and $\op_2$ denote completed read/write operations in an execution $\EX$, from processes $\pr_1,\pr_2\in\idSet$ respectively, such that $\op_1\bef\op_2$. If $\tg{\op_1}$ and $\tg{\op_2}$ are the 
	local tags at $\pr_1$ and $\pr_2$ after the completion of $\op_1$ and $\op_2$ respectively, then $\tg{\op_1} \leq \tg{\op_2}$; if $\op_1$ is a write operation then $\tg{\op_1}  < \tg{\op_2}$.
\end{lemma}

\begin{proof}
		Let $\tup{\tg{\op_1}, v_{\op_1}}$ be the pair passed to the last $\act{put-data}$ action of $\op_1$.
		Also, let $\state_2$ be the state in $\EX$ that follows the completion of the first $\act{read-config}$ action 
		during $\op_2$.
		Notice that $\op_2$ executes a loop after the first $\act{read-config}$ operation 
		and performs $c.\act{get-data}$ (if $\op_2$ is a read) or $c.\act{get-tag}$ (if $\op_2$
		is a write) from all $c =\config{\cvec{\pr_2}{\state_2}[i]}$, for $\mu(\cvec{\pr_2}{\state_2})\leq i \leq \nu(\cvec{\pr_2}{\state_2})$. By Lemma \ref{lem6}, there exists a $c'.\act{put-data}(\tup{\tg{},v})$ action by 
		some operation $\op'$ on some configuration $c'=\config{\cvec{\pr_2}{\state_2}[j]}$, for 
		$\mu(\cvec{\pr_2}{\state_2})\leq j \leq \nu(\cvec{\pr_2}{\state_2})$, that completes in some state 
		$\state'$ that appears before $\state_2$ in $\EX$. Thus, the $\act{get-data}$ or $\act{get-tag}$ 
		invoked by $\pr_2$ on $\config{\cvec{\pr_2}{\state_2}[j]}$, occurs after state $\state_2$ and thus
		after $\state'$. Since the DAP primitives used satisfy \textbf{C1} and \textbf{C2} of 
		Property \ref{property:dap}, then the $\act{get-tag}$ action will return a tag $\tg{\op_2}'$ 
		or a $\act{get-data}$ action will return a pair $\tup{\tg{\op_2}', v_{\op_2}'}$, with $\tg{\op_2}'\geq\tg{}$.
		As $\pr_2$ gets the maximum of all the tags returned, then by the end of the loop 
		$\pr_2$ will retrieve a tag $\tg{max}\geq\tg{\op_2}'\geq\tg{}\geq\tg{\op_1}$.
		
		If now $\op_2$ is a read, it returns $\tup{\tg{max}, v_{max}}$ after propagating that value to 
		the last discovered configuration. Thus, $\tg{\op_2}\geq \tg{\op_1}$. If however $\op_2$ is 
		a write, then before propagating the new value the writer increments the maximum timestamp
		discovered (Line A\ref{algo:writer}:\ref{line:writer:increment}) generating a tag $\tg{\op_2}>\tg{max}$.
		Therefore the operation $\op_2$ propagates a tag $\tg{\op_2}>\tg{\op_1}$ in this case.
\end{proof}

%\subsection*{ \ref{ssec:liveness} \ares{} Liveness}
%\input{ssec-correct-liveness.tex}
And the main result of this section follows:

\begin{theorem}[Atomicity]
	In  any execution $\EX$ of \ares{}, if in every configuration $c\in\gseq$,
	$\dagetdata{c}$, $\daputdata{c}{}$, and $\dagettag{c}$
	 %the DAP primitives  
	 satisfy Property~\ref{property:dap}, then ~\ares{} satisfy atomicity.
	%, given that the 
	%$\act{get-data}$, $\act{get-tag}$, and $\act{put-data}$ primitives used satisfy properties
	%\textbf{C1} and \textbf{C2} of Definition \ref{def:consistency}.
\end{theorem}

%In  \ares{},  each configuration 
%may implement the DAPs in a different way as stated below:
As algorithm \ares{} handles each configuration separately, then we can observe that the algorithm
may utilize a different mechanism for the put and get primitives in each configuration. So the 
following remark:

\begin{remark}
	Algorithm \ares{} satisfies atomicity even when the implementaton of the  DAPs in two 
	different configurations $c_1$ and $c_2$ are not the same, given that the $c_i.\act{get-tag}$,
	$c_i.\act{get-data}$, and the $c_i.\act{put-data}$ primitives 
	in each $c_i$ satisfy Property~\ref{property:dap}.  
\end{remark}
