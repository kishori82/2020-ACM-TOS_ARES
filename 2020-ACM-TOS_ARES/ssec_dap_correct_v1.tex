%\vspace{-1.em}
\myparagraph{Correctness.} 
In this section we are concerned with only one configuration $c$, consisting of a set of servers 
%$\mathcal{S}$
$\servers{c}$.
%, and a set of reader and writer clients $\mathcal{R}$ and $\mathcal{W}$, respectively. In other words, 
%in such static system the sets $\mathcal{S}$, $\mathcal{R}$ and $\mathcal{W}$ are fixed, and 
We assume that at most $f \leq \frac{n-k}{2}$ servers from $\servers{c}$ may crash.  
Lemma~\ref{casflex:data-access:consistent} states that the DAP implementation 
 satisfies the  consistency properties Property~\ref{property:dap}  which will be used to 
%of \treas{}, \nn{and in turn by Theorem \ref{atomicity:A1}} these 
imply the atomicity of the \ares{} algorithm. 
%which implies the atomicity city properties and consequently the
%atomicity property 
%(Theorem~\ref{thm:atomicity_radonc}).			
%\myparagraph{Liveness and Safety Conditions.}\blue{
%The \treas{} algorithm we present satisfy \myemph{wait-free termination} (Liveness) and \myemph{atomicity} (Safety).
%}
	%Due to lack of space the proof of the following Theorem is produced in the Appendix.	
\label{sec:primitives}

%
% 
% This abstraction enables us to prove the safety and liveness properties of such algorithms based on the properties of these primitives. 
% This abstraction servers us a two-fold 
% purpose: $(i)$ by expressing several atomicity emulation algorithm in terms of the primitives allows us to prove safety and liveness based on their properties $(iii)$ shows how such algorithms can be adopted to our ARES algorithm and prove their safety and liveness without; and $(iii)$ exposes the intuition that the underlying atomicity algorithm can  be different from configuration to configuration.
% For version control of the  object values  we use tags.  
% 
 
 
 %Let $<_\tau$ and $\leq_\tau$ be the appropriate comparison relationships used by any algorithm 
 %that utilizes logical timestamps. Then 
 %atomicity properties can be expressed in terms of the tags written and returned by write and read 
 %operations respectively. 
 %For a write operation $\wrt$ we denote by $\tg{\wrt}$ the tag that is 
 %used by $\wrt$ and for a read $\rd$ we denote by $\tg{\rd}$ the tag that is returned by $\rd$
 %\footnote{Note that the values written or returned by write of read operations can be mapped easily  
 %to the tags they write or return.}.	The partial ordering among the  operations  can then be induced from the partial ordering among the tags. 
 %using  tags in the following way: (i) for any two write 
 %operations $\wrt_1$, $\wrt_2$, if  $\wrt_1\prec\wrt_2$, then $\tg{\wrt_1}<_\tau\tg{\wrt_2}$,
 %(ii) For any operation $\op_1$,  and any read operation $\rd_2$, if $\op_1\prec\rd_2$, then
 %$\tg{\op_1}\leq_\tau\tg{\rd_2}$.

\proofremove{
 \begin{proof}
 We  prove the atomicity by proving properties $P1$, $P2$ and $P3$ appearing in Lemma \ref{XXX} for any execution of the algorithm.
					
	\emph{Property $P1$}: Consider two operations $\phi$ and $\pi$ such that $\phi$ completes before $\pi$ is invoked. 
	We need to show that it cannot be  the case that $\pi \prec \phi$. We break our analysis into the following four cases:

	Case $(a)$: {\em Both $\phi$ and $\pi$ are writes}. The $\daputdata{c}{*}$ of $\phi$ completes before 
	$\pi$ is invoked. 
	%which implies that by well-formedness 
	By property $C1$ the tag $\tg{\pi}$ returned by the $\dagetdata{c}$ at $\pi$ is 
	at least as large as $\tg{\phi}$. Now, 
	%since $\tg{\pi}$ is larger than $t_{\phi}$, by the steps of 
	since $\tg{\pi}$ is incremented by the write operation then $\pi$ puts a tag $\tg{\pi}'$ such that
	$\tg{\phi} < \tg{\pi}'$ and hence we cannot have $\pi \prec \phi$.
	
	Case $(b)$: {\em $\phi$ is a write and  $\pi$ is a read}. In execution $\EX$ since 
$\daputdata{c} {\tup{t_{\phi}, *}}$ of $\phi$ completes 
	before the $\dagetdata{c}$ of $\pi$ is invoked, by 
	%the well-formedness 
	property $C1$ the tag $\tg{\pi}$ obtained from the above
	$\dagetdata{c}$ is at least as large as $\tg{\phi}$. Now $\tg{\phi} \leq \tg{\pi}$ implies that we cannot have $\pi \prec \phi$.
	
	Case $(c)$: {\em $\phi$ is a read and  $\pi$ is a write}.  Let the id of the writer that invokes $\pi$ we $w_{\pi}$.  
	The 
$\daputdata{c}{\tup{\tg{\phi}, *}}$  call of $\phi$ completes
	before  $\dagettag{c}$ of $\pi$ is initiated. Therefore, by 
	%the well-formedness 
	property $C1$ %of data-primitives the above 
	$\act{get-tag}(c)$ returns $\tg{}$ such that, $\tg{\phi} \leq \tg{}$. Since $\tg{\pi}$ is equal to $(\tg{}.z + 1, w_{\pi})$ 
	by design of the algorithm, hence $\tg{\pi} > \tg{\phi}$ and we cannot have $\pi \prec \phi$.
	
	Case $(d)$: {\em Both $\phi$ and $\pi$ are reads}. In execution $\EX$  
the $\daputdata{c}{\tup{t_{\phi}, *}}$ is executed as a part of $\phi$ and 
	completes before $\dagetdata{c}$ is called in $\pi$. By 
	%the well-formedness
	 property $C1$ of the data-primitives, 
	we have $\tg{\phi} \leq \tg{\pi}$ and hence we cannot have $\pi \prec \phi$.
	
	\emph{Property $P2$}: Note that because $\tsSet$ is well-ordered we can show that this property by first showing that
	every write has a unique tag. This means any two pair of writes can be ordered. Now, a read can be ordered . Note that 
	a read can be ordered w.r.t. to any write operation trivially if the respective tags are different, and by definition, if the 
	tags are equal the write is ordered before the read.
	
	Now observe that two tags generated from two write operations from different writers are necessarily distinct because of the 
	id component of the tag. Now if the operations, say $\phi$ and $\pi$ are writes  from the same writer then by 
	well-formedness property the second operation is invoked after the first completes, say without loss of generality $\phi$ completes before 
	$\pi$ is invoked.   In that case the integer part of the tag of $\pi$ is higher 
	%because the well-formedness 
	by property  $C1$, and since the $\dagettag{c}$  is followed by $\daputdata{c}{*}$. Hence $\pi$ is ordered after $\phi$. 
	
	\emph{Property $P3$}:  This is clear because the tag of a reader is defined by the tag of the value it returns by property (b).
	Therefore, the reader's immediate previous value it returns. On the other hand if  does 
	note return any write operation's value it must return $v_0$.
 \end{proof}
}



						
 \begin{theorem}[Safety]\label{casflex:data-access:consistent}
Let $\Pi$ a set of complete DAP operations of Algorithm \ref{fig:casopt} in a configuration $c\in\confSet$,
$\act{c.get-tag}$, $\act{c.get-data}$ and $\act{c.put-data}$,
of an execution $\EX$. Then, every pair of operations $\phi,\op\in\Pi$ satisfy Property \ref{property:dap}.
% The data-access primitives, i.e., $\act{get-tag}$, $\act{get-data}$ and $\act{put-data}$ primitives implemented in any configuration  $c$
% in this section satisfy Property~\ref{property:dap}.
\end{theorem}


\proofremove{
\begin{proof}
As mentioned above we are concerned with only configuration $c$, and therefore, in our proofs we will be concerned with only one
configuration. Let $\alpha$ be some execution of \treas{}, then we consider two cases for $\pi$ for proving property $C1$:  $\pi$ is a  $\act{get-tag}$ operation, or $\pi$ is a $\act{get-data}$ primitive. 

 %\item[ C1 ]  If $\phi$ is a  $\daputdata{c}{\tup{\tg{\phi}, v_\phi}}$, for $c \in \confSet$, $\tg{1} \in\tsSet$ and $v_1 \in \valSet$,
 %and $\pi$ is a $\dagettag{c}$ (or a $\dagetdata{c}$) 

 %that returns $\tg{\pi} \in \tsSet$ (or $\tup{\tg{\pi}, v_{\pi}} \in \tsSet \times \valSet$) and $\phi$ completes before $\pi$ in $\EX$, then $\tg{\pi} \geq \tg{\phi}$.
Case $(a)$: $\phi$ is   $\daputdata{c}{\tup{\tg{\phi}, v_\phi}}$ and  $\pi$ is a $\dagettag{c}$ returns $\tg{\pi} \in \tsSet$. Let $c_{\phi}$ and $c_{\pi}$ denote the clients that invokes $\phi$ and $\pi$ in $\alpha$. Let $S_{\phi} \subset \mathcal{S}$ denote the set of $\left\lceil \frac{n+k}{2} \right \rceil$ servers that responds to $c_{\phi}$, during $\phi$. Denote by $S_{\pi}$ the set of $\left\lceil \frac{n+k}{2} \right \rceil$ servers that responds to $c_{\pi}$, during $\pi$.  Let $T_1$ be a point in execution $\alpha$ 
after the completion of $\phi$ and before the invocation of $\pi$. Because $\pi$ is invoked after $T_1$, therefore, at $T_1$ each of the servers in $S_{\phi}$ contains $t_{\phi}$ in its $List$ variable. Note that, once a tag is added to $List$, it is never removed. Therefore, during $\pi$, any server in $S_{\phi}\cap S_{\pi}$ responds with $List$ containing $t_{\phi}$ to $c_{\pi}$. Note that since  $|S_{\sigma^*}| = |S_{\pi}| =\left\lceil \frac{n+k}{2} \right \rceil $ implies
				 $| S_{\sigma^*} \cap S_{\pi} | \geq k$, and hence $t^{dec}_{max}$ at $c_{\pi}$, during $\pi$ is at least as large as $t_{\phi}$, i.e., $t_{\pi} \geq t_{\phi}$. Therefore, it suffices to to prove our claim with respect to the tags and the decodability of  its corresponding value.


Case $(b)$: $\phi$ is   $\daputdata{c}{\tup{\tg{\phi}, v_\phi}}$ and  $\pi$ is a $\dagetdata{c}$ returns $\tup{\tg{\pi}, v_{\pi}} \in \tsSet \times \valSet$. 
As above, let $c_{\phi}$ and $c_{\pi}$ be the clients that invokes $\phi$ and 
$\pi$. Let $S_{\phi}$ and $S_{\pi}$ be the set of servers that responds to $c_{\phi}$ and $c_{\pi}$, respectively. Arguing as above, 
 $| S_{\sigma^*} \cap S_{\pi} | \geq k$ and every server in  $S_{\phi} \cap S_{\pi} $ sends $t_{\phi}$ in response to $c_{\phi}$, during 
 $\pi$, in their $List$'s and hence $t_{\phi} \in Tags_{*}^{\geq k}$. Now, because $\pi$ completes in $\alpha$, hence we have 
 $t^*_{max} = t^{dec}_{max}$. Note that $\max Tags_{*}^{\geq k} \geq \max Tags_{dec}^{\geq k}$ so 
  $t_{\pi} \geq \max Tags_{dec}^{\geq k} = \max Tags_{*}^{\geq k} \geq t_{\phi}$. Note that each tag is always associated with 
  its corresponding value $v_{\pi}$, or the corresponding coded elements $\Phi_s(v_{\pi})$ for $s \in \mathcal{S}$.

Next, we prove the $C2$ property of DAP for the \treas{} algorithm. Note that the initial values of the $List$ variable in each servers $s$ in $\mathcal{S}$ is 
$\{ (t_0, \Phi_s(v_{\pi}) )\}$. Moreover, from an inspection of the steps of the algorithm, new tags in the $List$ variable of any servers of any servers is introduced via $\act{put-data}$ operation. Since $t_{\pi}$ is returned by a $\act{get-tag}$ or 
$\act{get-data}$ operation then it must be that either $t_{\pi}=t_0$ or $t_{\pi} > t_0$. In the case where $t_{\pi} = t_0$ then we have nothing to prove. If $t_{\pi} > t_0$ then there must be a $\act{put-data}(t_{\pi}, v_{\pi})$ operation $\phi$. To show that for every $\pi$ it cannot be that $\phi$ completes before $\pi$, we adopt by a contradiction. Suppose for every $\pi$, $\phi$ completes before $\pi$ begins, then clearly $t_{\pi}$ cannot be returned $\phi$, a contradiction.
\end{proof}
}			
	\remove{
				\begin{theorem}[Atomicity]  \label{thm:atomicity_radonc}
					Any well-formed and fair execution of \treas{},  is atomic.
				\end{theorem}
		}
	\myparagraph{Liveness.} \label{sec:treas_liveness}
    To reason about the liveness of the proposed DAPs, we define a concurrency parameter $\delta$ which  captures all the  $\act{put-data}$ operations that overlap with the $\act{get-data}$, until the time the client has all data needed to attempt decoding a value. However, we ignore those $\act{put-data}$ operations that might have started in the past, and never completed yet, if their tags are less than that of any $\act{put-data}$ that completed before the  $\act{get-data}$  started. This allows us to ignore $\act{put-data}$ operations due to failed clients, while counting concurrency, as long as the failed $\act{put-data}$ operations are followed by a successful $\act{put-data}$ that completed before the $\act{get-data}$ started. 				
\kmk{In order to define the amount of concurrency  in  our specific implementation of the DAPs presented in this section the}  following definition captures the $\act{put-data}$ operations that overlap with the $\act{get-data}$, until  the client has all data required to  decode the value.
				
\begin{definition}[Valid $\act{get-data}$ operations]
A $\act{get-data}$  operation $\pi$ from a process $p$ is \myemph{valid}  if 
%the associated client 
$p$ does not crash until the reception of $\left\lceil \frac{n+k}{2} \right\rceil$ responses during the{\GetData} phase. 
\end{definition}
					
				
				\begin{definition}[$\act{put-data}$ concurrent with a valid $\act{get-data}$] \label{defn:concurrent}
					Consider a valid $\act{get-data}$ operation $\pi$ from a process $p$. 
					Let $T_1$ denote the point of initiation of $\pi$. For $\pi$, let $T_2$ denote the earliest point of time during the execution when $p$ 
					%the associated client 
					receives all the $\left\lceil \frac{n+k}{2} \right\rceil$ responses.
					% For a valid repair,  let $T_2$ denote the point of time during the execution when the repair completes, and takes the associated server back to the active state. 
					Consider the set $\Sigma = \{ \phi: \phi$ is any $\act{put-data}$ operation that completes before $\pi \text{ is initiated} \}$, and let $\phi^* = \arg\max_{\phi \in \Sigma}tag(\phi)$. Next, consider the set $\Lambda = \{\lambda:  \lambda$  is any $\act{put-data}$ operation that starts before $T_2 \text{ such that } tag(\lambda) > tag(\phi^*)\}$. We define the number of $\act{put-data}$ concurrent with the valid $\act{get-data}$  $\pi$ to be the cardinality of the set $\Lambda$.
				\end{definition}
							
Termination (and hence liveness)  of the DAPs is guaranteed in an execution $\EX$, provided that a process 
	no more than $f$ servers in $\servers{c}$ crash, and no more that $\delta$ $\act{put-data}$ may be concurrent at any point in $\EX$. 
	%in  property of an algorithm,  we mean that 
	If the failure model is satisfied, then any operation invoked by a non-faulty client will collect the necessary replies
	% process terminates  
	independently of the progress of any other client process in the system. Preserving $\delta$ on the other hand,
	ensures that any operation will be able to decode a written value. These are captured in the following theorem:

				\begin{theorem}[Liveness]  \label{thm:liveness_radonc}
					Let $\EX$ be well-formed and fair execution of DAPs, with an $[n, k]$ MDS code, 
					where $n$ is the number of servers out of which no more than $\frac{n-k}{2}$ may crash, 
					%and $k  > n/3$,
					 and $\delta$ be the maximum number of $\act{put-data}$ operations concurrent with any 
					 valid $\act{get-data}$ operation. 
					 Then any $\act{get-data}$ and $\act{put-data}$ operation $\op$ 
					 invoked by a process $\pr$  terminates in $\EX$ if $\pr$
					 does not crash between the invocation and response steps of $\op$.\vspace{-.5em}
				\end{theorem}
		\proofremove{		
				\begin{proof}
				Note that in the read and write operation the  $\act{get-tag}$ and $\act{put-data}$ operations initiated by any non-faulty client  always complete.
				Therefore, the liveness property with respect to any write operation is clear because it uses only  $\act{get-tag}$ and $\act{put-data}$ operations of the DAP. So, we focus on proving the liveness property of any read operation $\pi$, 
				specifically,   the  $\act{get-data}$ operation completes. Let $\alpha $ be and execution of \treas{} and let 
				$c_{\sigma^*}$ and $c_{\pi}$ be the clients that invokes the write operation $\sigma^*$ and 
				read operation $c_{\pi}$, respectively.
				
				Let $S_{\sigma^{*}}$ be the set of 
				$\left\lceil \frac{n+k}{2} \right \rceil$ servers that responds to 
				$c_{\sigma^*}$, in the $\act{put-data}$ operations, in $\sigma^*$.
				 Let $S_{\sigma^{\pi}}$ be the set of $\left\lceil \frac{n+k}{2} \right \rceil$ servers that responds to  $c_{\pi}$ during the  $\act{get-data}$ step of $\pi$. Note that in $\alpha$ at the point execution $T_1$, just before the execution of  $\pi$, none of the the write operations in 
				 $\Lambda$ is complete. Observe that,  by algorithm design, the coded-elements corresponding to  $t_{\sigma^*}$ are garbage-collected from the $List$ variable of a server only if more than $\delta$ higher tags are introduced by subsequent writes into the server.  Since the number of concurrent writes  $|\Lambda|$, s.t.  $\delta > | \Lambda |$ the corresponding value of tag $t_{\sigma^*}$ is not garbage collected in $\alpha$, at least until execution point $T_2$  in  any of the servers in $S_{\sigma^*}$.
				 
				 Therefore, during the execution fragment between the execution points $T_1$ and $T_2$ of the execution $\alpha$, the tag and coded-element pair is present in the $List$ variable of every in $S_{\sigma^*}$ that is active. As a result, the tag and coded-element pairs, $(t_{\sigma^*}, \Phi_s(v_{\sigma^*}))$ exists in the $List$ received from any
				  $s \in S_{\sigma^*} \cap S_{\pi}$ during operation $\pi$. Note that since $|S_{\sigma^*}| = |S_{\pi}| =\left\lceil \frac{n+k}{2} \right \rceil $ hence
				 $| S_{\sigma^*} \cap S_{\pi} | \geq k$ and hence 
				 $t_{\sigma^*} \in Tags_{dec}^{\geq k} $, the set of decodable tag, i.e., the value $v_{\sigma^*}$ can be decoded
				  by $c_{\pi}$ in $\pi$, which demonstrates that $Tags_{dec}^{\geq k}  \neq \emptyset$. Next we want to 
				  argue that 
				  $t_{max}^* = t_{max}^{dec}$ via a contradiction: we assume 
				  $ \max Tags_{*}^{\geq k}  >  \max Tags_{dec}^{\geq k}  $. Now, consider any tag $t$, which  exists due to our assumption,  such that, 
				  $t \in Tags_{*}^{\geq k} $,  $t \not\in Tags_{dec}^{\geq k} $ and $t > t_{max}^{dec}$.
			%	 
				 Let $S^k_{\pi} \subset S$ be any subset of $k$ servers that responds with $t^*_{max}$ in their $List$ variables to $c_{\pi}$. Note that since $k >  n/3$ hence $|S_{\sigma^*} \cap S_{\pi}|  \geq \left\lceil \frac{n+k}{2} \right \rceil +  \left\lceil \frac{n+1}{3} \right \rceil \geq 1$, i.e., $S_{\sigma^*} \cap S_{\pi} \neq \emptyset$. Then $t$ 
				 must be in some servers in $S_{\sigma^*}$ at $T_2$ and since $t > t_{max}^{dec} \geq t_{\sigma^*}$. 
				 Now since $|\Lambda| < \delta$ hence $(t, \bot)$ cannot be in any server at $T_2$  because there are not enough concurrent write operations (i.e., writes in $\Lambda$) to garbage-collect the coded-elements corresponding to tag $t$, which also holds  for tag  $t^{*}_{max}$. In that case, $t$ must be in $Tag_{dec}^{\geq k}$, a contradiction.
%
				\end{proof}
}