In this section, we prove that \ares{} correctly implements an atomic, read/write, shared storage service. 
%We also provide an analysis of its storage and communication costs, and the latency of read and write operations. 
The correctness of \ares{} highly depends on the way the configuration 
sequence is constructed at each client process.
Also, atomicity is ensured if the DAP implementation in each configuration $c_i$
%relies on the satisfaction of 
%hinges on 
satisfies Property~\ref{property:dap}.
Thus, we begin by showing some critical properties preserved by the reconfiguration service proposed in \ares{} in subsection \ref{sec:safety:recon}, and then we proof
the correctness of \ares{} in subsection \ref{sec:safety:atomic} 
when those properties hold and the DAPs used in 
each configuration satisfy Property \ref{property:dap}.

% of the  used in an execution of~\ares{}. 
%Due to lack of space proofs are omitted and can be found in the 
%% of the following Theorem is produced in the 
%extended version of the paper~\cite{ARES:Arxiv:2018}.	

We proceed by first introducing some definitions and notation, that we use in the proofs that follow. 
%then by presenting some properties that are satisfied 
%	by the reconfiguration service in any execution, and then we show that given these properties our algorithm satisfies 
%	the safety (atomicity) conditions. 

	\myparagraph{Notations and definitions.}
	For a server $s$, we use the notation $\atT{s.var}{\state}$ to refer to the value of the state variable $var$, in $s$, at a state $\state$ of an  execution $\EX$. 
	If server  $s$ crashes at a state $\state_f$ in an execution $\EX$ then $\atT{s.var}{\state}\triangleq\atT{s.var}{\state_f}$ for any state variable $var$ and for 
	any state $\state$ that appears after $\state_f$ in $\EX$. 
	%refers to the value of $v$ at $s$ at the state just before crashes. In other words, $\atT{s.v}{T}  \triangleq \atT{s.v}{\hat{T}}$, where $\hat{T}$ is the latest point in the execution, such that, $(a)$ $\hat{T} \leq T$ and $(b)$ $s$ is non-faulty.

    We define as the tag of a configuration $c$ the smallest tag among the maximum tags found in each quorum of $c$. This is essentially the smallest tag that an operation may witness when receiving replies from a single quorum in $c$. More formally:

	\begin{definition}[Tag of a configuration]  Let  $c \in \mathcal{C}$ be a configuration, $\state$ be a state in some execution $\EX$ then 
		we define the tag of $c$ at state $\state$ as  
		$ \atT{tag(c)}{\state} \triangleq \min_{Q \in \quorums{c}} \max_{s \in Q}~\atT{(s.tag}{\state}).$
		We  drop the suffix $|_\state$, and simply denote as $tag(c)$,  when the state  is clear from the context.
	\end{definition}
	Next we provide the notation to express the configuration sequence witnessed by a process $p$ in a state $\state$ (as $ \atT{\pr.cseq}{\state}$), the last finalized configuration in that sequence
	(as $\mu(\cvec{\pr}{\state})$), and the length of that sequence (as $\nu(\cvec{\pr}{\state})$). More formally:
	
	\begin{definition}
		Let $\sigma$ be any point in an execution of \ares{} and suppose we use the notation $\cvec{\pr}{\state}$ for $ \atT{\pr.cseq}{\state}$,  i.e., the $cseq$ variable at process $p$ at the state $\state$. %be the value of a configuration sequence vector at a process $\pr$ at some state  $\st$ in an execution $\EX$. 
		Then we define as $ \mu(\cvec{\pr}{\state})  \triangleq  \max\{ i : \cvec{\pr}{\state}[i].status = F\}$ 
		and $ \nu(\cvec{\pr}{\state}) \triangleq |\cvec{\pr}{\state}|$, where $|\cvec{\pr}{\state}|$ is the length of the  configuration vector 
		$\cvec{\pr}{\state}$. % that are not equal to $\bot$.  
	\end{definition}
	
	Last, we define the prefix operation on two configuration sequences. 
	
	\begin{definition} [Prefix order]
		Let $\mathbf{x}$ and $\mathbf{y}$ be any two configuration sequences. We say that $\mathbf{x}$ is a prefix of $\mathbf{y}$, denoted by 
		$\mathbf{x} \preceq_p  \mathbf{y}$, if $\config{\mathbf{x}[j]}=\config{\mathbf{y}[j]}$, for all $j$ such that $\mathbf{x}[j]\neq\bot$.
	\end{definition}

% \subsection{Safety (Property~\ref{property:dap})  proof of the DAP{s}}
% \label{sec:safety:daps}
% %\vspace{-1.em}
\myparagraph{Correctness.} 
In this section we are concerned with only one configuration $c$, consisting of a set of servers 
%$\mathcal{S}$
$\servers{c}$.
%, and a set of reader and writer clients $\mathcal{R}$ and $\mathcal{W}$, respectively. In other words, 
%in such static system the sets $\mathcal{S}$, $\mathcal{R}$ and $\mathcal{W}$ are fixed, and 
We assume that at most $f \leq \frac{n-k}{2}$ servers from $\servers{c}$ may crash.  
Lemma~\ref{casflex:data-access:consistent} states that the DAP implementation 
 satisfies the  consistency properties Property~\ref{property:dap}  which will be used to 
%of \treas{}, \nn{and in turn by Theorem \ref{atomicity:A1}} these 
imply the atomicity of the \ares{} algorithm. 
%which implies the atomicity city properties and consequently the
%atomicity property 
%(Theorem~\ref{thm:atomicity_radonc}).			
%\myparagraph{Liveness and Safety Conditions.}\blue{
%The \treas{} algorithm we present satisfy \myemph{wait-free termination} (Liveness) and \myemph{atomicity} (Safety).
%}
	%Due to lack of space the proof of the following Theorem is produced in the Appendix.	
\label{sec:primitives}

%
% 
% This abstraction enables us to prove the safety and liveness properties of such algorithms based on the properties of these primitives. 
% This abstraction servers us a two-fold 
% purpose: $(i)$ by expressing several atomicity emulation algorithm in terms of the primitives allows us to prove safety and liveness based on their properties $(iii)$ shows how such algorithms can be adopted to our ARES algorithm and prove their safety and liveness without; and $(iii)$ exposes the intuition that the underlying atomicity algorithm can  be different from configuration to configuration.
% For version control of the  object values  we use tags.  
% 
 
 
 %Let $<_\tau$ and $\leq_\tau$ be the appropriate comparison relationships used by any algorithm 
 %that utilizes logical timestamps. Then 
 %atomicity properties can be expressed in terms of the tags written and returned by write and read 
 %operations respectively. 
 %For a write operation $\wrt$ we denote by $\tg{\wrt}$ the tag that is 
 %used by $\wrt$ and for a read $\rd$ we denote by $\tg{\rd}$ the tag that is returned by $\rd$
 %\footnote{Note that the values written or returned by write of read operations can be mapped easily  
 %to the tags they write or return.}.	The partial ordering among the  operations  can then be induced from the partial ordering among the tags. 
 %using  tags in the following way: (i) for any two write 
 %operations $\wrt_1$, $\wrt_2$, if  $\wrt_1\prec\wrt_2$, then $\tg{\wrt_1}<_\tau\tg{\wrt_2}$,
 %(ii) For any operation $\op_1$,  and any read operation $\rd_2$, if $\op_1\prec\rd_2$, then
 %$\tg{\op_1}\leq_\tau\tg{\rd_2}$.

\proofremove{
 \begin{proof}
 We  prove the atomicity by proving properties $P1$, $P2$ and $P3$ appearing in Lemma \ref{XXX} for any execution of the algorithm.
					
	\emph{Property $P1$}: Consider two operations $\phi$ and $\pi$ such that $\phi$ completes before $\pi$ is invoked. 
	We need to show that it cannot be  the case that $\pi \prec \phi$. We break our analysis into the following four cases:

	Case $(a)$: {\em Both $\phi$ and $\pi$ are writes}. The $\daputdata{c}{*}$ of $\phi$ completes before 
	$\pi$ is invoked. 
	%which implies that by well-formedness 
	By property $C1$ the tag $\tg{\pi}$ returned by the $\dagetdata{c}$ at $\pi$ is 
	at least as large as $\tg{\phi}$. Now, 
	%since $\tg{\pi}$ is larger than $t_{\phi}$, by the steps of 
	since $\tg{\pi}$ is incremented by the write operation then $\pi$ puts a tag $\tg{\pi}'$ such that
	$\tg{\phi} < \tg{\pi}'$ and hence we cannot have $\pi \prec \phi$.
	
	Case $(b)$: {\em $\phi$ is a write and  $\pi$ is a read}. In execution $\EX$ since 
$\daputdata{c} {\tup{t_{\phi}, *}}$ of $\phi$ completes 
	before the $\dagetdata{c}$ of $\pi$ is invoked, by 
	%the well-formedness 
	property $C1$ the tag $\tg{\pi}$ obtained from the above
	$\dagetdata{c}$ is at least as large as $\tg{\phi}$. Now $\tg{\phi} \leq \tg{\pi}$ implies that we cannot have $\pi \prec \phi$.
	
	Case $(c)$: {\em $\phi$ is a read and  $\pi$ is a write}.  Let the id of the writer that invokes $\pi$ we $w_{\pi}$.  
	The 
$\daputdata{c}{\tup{\tg{\phi}, *}}$  call of $\phi$ completes
	before  $\dagettag{c}$ of $\pi$ is initiated. Therefore, by 
	%the well-formedness 
	property $C1$ %of data-primitives the above 
	$\act{get-tag}(c)$ returns $\tg{}$ such that, $\tg{\phi} \leq \tg{}$. Since $\tg{\pi}$ is equal to $(\tg{}.z + 1, w_{\pi})$ 
	by design of the algorithm, hence $\tg{\pi} > \tg{\phi}$ and we cannot have $\pi \prec \phi$.
	
	Case $(d)$: {\em Both $\phi$ and $\pi$ are reads}. In execution $\EX$  
the $\daputdata{c}{\tup{t_{\phi}, *}}$ is executed as a part of $\phi$ and 
	completes before $\dagetdata{c}$ is called in $\pi$. By 
	%the well-formedness
	 property $C1$ of the data-primitives, 
	we have $\tg{\phi} \leq \tg{\pi}$ and hence we cannot have $\pi \prec \phi$.
	
	\emph{Property $P2$}: Note that because $\tsSet$ is well-ordered we can show that this property by first showing that
	every write has a unique tag. This means any two pair of writes can be ordered. Now, a read can be ordered . Note that 
	a read can be ordered w.r.t. to any write operation trivially if the respective tags are different, and by definition, if the 
	tags are equal the write is ordered before the read.
	
	Now observe that two tags generated from two write operations from different writers are necessarily distinct because of the 
	id component of the tag. Now if the operations, say $\phi$ and $\pi$ are writes  from the same writer then by 
	well-formedness property the second operation is invoked after the first completes, say without loss of generality $\phi$ completes before 
	$\pi$ is invoked.   In that case the integer part of the tag of $\pi$ is higher 
	%because the well-formedness 
	by property  $C1$, and since the $\dagettag{c}$  is followed by $\daputdata{c}{*}$. Hence $\pi$ is ordered after $\phi$. 
	
	\emph{Property $P3$}:  This is clear because the tag of a reader is defined by the tag of the value it returns by property (b).
	Therefore, the reader's immediate previous value it returns. On the other hand if  does 
	note return any write operation's value it must return $v_0$.
 \end{proof}
}



						
 \begin{theorem}[Safety]\label{casflex:data-access:consistent}
Let $\Pi$ a set of complete DAP operations of Algorithm \ref{fig:casopt} in a configuration $c\in\confSet$,
$\act{c.get-tag}$, $\act{c.get-data}$ and $\act{c.put-data}$,
of an execution $\EX$. Then, every pair of operations $\phi,\op\in\Pi$ satisfy Property \ref{property:dap}.
% The data-access primitives, i.e., $\act{get-tag}$, $\act{get-data}$ and $\act{put-data}$ primitives implemented in any configuration  $c$
% in this section satisfy Property~\ref{property:dap}.
\end{theorem}


\proofremove{
\begin{proof}
As mentioned above we are concerned with only configuration $c$, and therefore, in our proofs we will be concerned with only one
configuration. Let $\alpha$ be some execution of \treas{}, then we consider two cases for $\pi$ for proving property $C1$:  $\pi$ is a  $\act{get-tag}$ operation, or $\pi$ is a $\act{get-data}$ primitive. 

 %\item[ C1 ]  If $\phi$ is a  $\daputdata{c}{\tup{\tg{\phi}, v_\phi}}$, for $c \in \confSet$, $\tg{1} \in\tsSet$ and $v_1 \in \valSet$,
 %and $\pi$ is a $\dagettag{c}$ (or a $\dagetdata{c}$) 

 %that returns $\tg{\pi} \in \tsSet$ (or $\tup{\tg{\pi}, v_{\pi}} \in \tsSet \times \valSet$) and $\phi$ completes before $\pi$ in $\EX$, then $\tg{\pi} \geq \tg{\phi}$.
Case $(a)$: $\phi$ is   $\daputdata{c}{\tup{\tg{\phi}, v_\phi}}$ and  $\pi$ is a $\dagettag{c}$ returns $\tg{\pi} \in \tsSet$. Let $c_{\phi}$ and $c_{\pi}$ denote the clients that invokes $\phi$ and $\pi$ in $\alpha$. Let $S_{\phi} \subset \mathcal{S}$ denote the set of $\left\lceil \frac{n+k}{2} \right \rceil$ servers that responds to $c_{\phi}$, during $\phi$. Denote by $S_{\pi}$ the set of $\left\lceil \frac{n+k}{2} \right \rceil$ servers that responds to $c_{\pi}$, during $\pi$.  Let $T_1$ be a point in execution $\alpha$ 
after the completion of $\phi$ and before the invocation of $\pi$. Because $\pi$ is invoked after $T_1$, therefore, at $T_1$ each of the servers in $S_{\phi}$ contains $t_{\phi}$ in its $List$ variable. Note that, once a tag is added to $List$, it is never removed. Therefore, during $\pi$, any server in $S_{\phi}\cap S_{\pi}$ responds with $List$ containing $t_{\phi}$ to $c_{\pi}$. Note that since  $|S_{\sigma^*}| = |S_{\pi}| =\left\lceil \frac{n+k}{2} \right \rceil $ implies
				 $| S_{\sigma^*} \cap S_{\pi} | \geq k$, and hence $t^{dec}_{max}$ at $c_{\pi}$, during $\pi$ is at least as large as $t_{\phi}$, i.e., $t_{\pi} \geq t_{\phi}$. Therefore, it suffices to to prove our claim with respect to the tags and the decodability of  its corresponding value.


Case $(b)$: $\phi$ is   $\daputdata{c}{\tup{\tg{\phi}, v_\phi}}$ and  $\pi$ is a $\dagetdata{c}$ returns $\tup{\tg{\pi}, v_{\pi}} \in \tsSet \times \valSet$. 
As above, let $c_{\phi}$ and $c_{\pi}$ be the clients that invokes $\phi$ and 
$\pi$. Let $S_{\phi}$ and $S_{\pi}$ be the set of servers that responds to $c_{\phi}$ and $c_{\pi}$, respectively. Arguing as above, 
 $| S_{\sigma^*} \cap S_{\pi} | \geq k$ and every server in  $S_{\phi} \cap S_{\pi} $ sends $t_{\phi}$ in response to $c_{\phi}$, during 
 $\pi$, in their $List$'s and hence $t_{\phi} \in Tags_{*}^{\geq k}$. Now, because $\pi$ completes in $\alpha$, hence we have 
 $t^*_{max} = t^{dec}_{max}$. Note that $\max Tags_{*}^{\geq k} \geq \max Tags_{dec}^{\geq k}$ so 
  $t_{\pi} \geq \max Tags_{dec}^{\geq k} = \max Tags_{*}^{\geq k} \geq t_{\phi}$. Note that each tag is always associated with 
  its corresponding value $v_{\pi}$, or the corresponding coded elements $\Phi_s(v_{\pi})$ for $s \in \mathcal{S}$.

Next, we prove the $C2$ property of DAP for the \treas{} algorithm. Note that the initial values of the $List$ variable in each servers $s$ in $\mathcal{S}$ is 
$\{ (t_0, \Phi_s(v_{\pi}) )\}$. Moreover, from an inspection of the steps of the algorithm, new tags in the $List$ variable of any servers of any servers is introduced via $\act{put-data}$ operation. Since $t_{\pi}$ is returned by a $\act{get-tag}$ or 
$\act{get-data}$ operation then it must be that either $t_{\pi}=t_0$ or $t_{\pi} > t_0$. In the case where $t_{\pi} = t_0$ then we have nothing to prove. If $t_{\pi} > t_0$ then there must be a $\act{put-data}(t_{\pi}, v_{\pi})$ operation $\phi$. To show that for every $\pi$ it cannot be that $\phi$ completes before $\pi$, we adopt by a contradiction. Suppose for every $\pi$, $\phi$ completes before $\pi$ begins, then clearly $t_{\pi}$ cannot be returned $\phi$, a contradiction.
\end{proof}
}			
	\remove{
				\begin{theorem}[Atomicity]  \label{thm:atomicity_radonc}
					Any well-formed and fair execution of \treas{},  is atomic.
				\end{theorem}
		}
	\myparagraph{Liveness.} \label{sec:treas_liveness}
    To reason about the liveness of the proposed DAPs, we define a concurrency parameter $\delta$ which  captures all the  $\act{put-data}$ operations that overlap with the $\act{get-data}$, until the time the client has all data needed to attempt decoding a value. However, we ignore those $\act{put-data}$ operations that might have started in the past, and never completed yet, if their tags are less than that of any $\act{put-data}$ that completed before the  $\act{get-data}$  started. This allows us to ignore $\act{put-data}$ operations due to failed clients, while counting concurrency, as long as the failed $\act{put-data}$ operations are followed by a successful $\act{put-data}$ that completed before the $\act{get-data}$ started. 				
\kmk{In order to define the amount of concurrency  in  our specific implementation of the DAPs presented in this section the}  following definition captures the $\act{put-data}$ operations that overlap with the $\act{get-data}$, until  the client has all data required to  decode the value.
				
\begin{definition}[Valid $\act{get-data}$ operations]
A $\act{get-data}$  operation $\pi$ from a process $p$ is \myemph{valid}  if 
%the associated client 
$p$ does not crash until the reception of $\left\lceil \frac{n+k}{2} \right\rceil$ responses during the{\GetData} phase. 
\end{definition}
					
				
				\begin{definition}[$\act{put-data}$ concurrent with a valid $\act{get-data}$] \label{defn:concurrent}
					Consider a valid $\act{get-data}$ operation $\pi$ from a process $p$. 
					Let $T_1$ denote the point of initiation of $\pi$. For $\pi$, let $T_2$ denote the earliest point of time during the execution when $p$ 
					%the associated client 
					receives all the $\left\lceil \frac{n+k}{2} \right\rceil$ responses.
					% For a valid repair,  let $T_2$ denote the point of time during the execution when the repair completes, and takes the associated server back to the active state. 
					Consider the set $\Sigma = \{ \phi: \phi$ is any $\act{put-data}$ operation that completes before $\pi \text{ is initiated} \}$, and let $\phi^* = \arg\max_{\phi \in \Sigma}tag(\phi)$. Next, consider the set $\Lambda = \{\lambda:  \lambda$  is any $\act{put-data}$ operation that starts before $T_2 \text{ such that } tag(\lambda) > tag(\phi^*)\}$. We define the number of $\act{put-data}$ concurrent with the valid $\act{get-data}$  $\pi$ to be the cardinality of the set $\Lambda$.
				\end{definition}
							
Termination (and hence liveness)  of the DAPs is guaranteed in an execution $\EX$, provided that a process 
	no more than $f$ servers in $\servers{c}$ crash, and no more that $\delta$ $\act{put-data}$ may be concurrent at any point in $\EX$. 
	%in  property of an algorithm,  we mean that 
	If the failure model is satisfied, then any operation invoked by a non-faulty client will collect the necessary replies
	% process terminates  
	independently of the progress of any other client process in the system. Preserving $\delta$ on the other hand,
	ensures that any operation will be able to decode a written value. These are captured in the following theorem:

				\begin{theorem}[Liveness]  \label{thm:liveness_radonc}
					Let $\EX$ be well-formed and fair execution of DAPs, with an $[n, k]$ MDS code, 
					where $n$ is the number of servers out of which no more than $\frac{n-k}{2}$ may crash, 
					%and $k  > n/3$,
					 and $\delta$ be the maximum number of $\act{put-data}$ operations concurrent with any 
					 valid $\act{get-data}$ operation. 
					 Then any $\act{get-data}$ and $\act{put-data}$ operation $\op$ 
					 invoked by a process $\pr$  terminates in $\EX$ if $\pr$
					 does not crash between the invocation and response steps of $\op$.\vspace{-.5em}
				\end{theorem}
		\proofremove{		
				\begin{proof}
				Note that in the read and write operation the  $\act{get-tag}$ and $\act{put-data}$ operations initiated by any non-faulty client  always complete.
				Therefore, the liveness property with respect to any write operation is clear because it uses only  $\act{get-tag}$ and $\act{put-data}$ operations of the DAP. So, we focus on proving the liveness property of any read operation $\pi$, 
				specifically,   the  $\act{get-data}$ operation completes. Let $\alpha $ be and execution of \treas{} and let 
				$c_{\sigma^*}$ and $c_{\pi}$ be the clients that invokes the write operation $\sigma^*$ and 
				read operation $c_{\pi}$, respectively.
				
				Let $S_{\sigma^{*}}$ be the set of 
				$\left\lceil \frac{n+k}{2} \right \rceil$ servers that responds to 
				$c_{\sigma^*}$, in the $\act{put-data}$ operations, in $\sigma^*$.
				 Let $S_{\sigma^{\pi}}$ be the set of $\left\lceil \frac{n+k}{2} \right \rceil$ servers that responds to  $c_{\pi}$ during the  $\act{get-data}$ step of $\pi$. Note that in $\alpha$ at the point execution $T_1$, just before the execution of  $\pi$, none of the the write operations in 
				 $\Lambda$ is complete. Observe that,  by algorithm design, the coded-elements corresponding to  $t_{\sigma^*}$ are garbage-collected from the $List$ variable of a server only if more than $\delta$ higher tags are introduced by subsequent writes into the server.  Since the number of concurrent writes  $|\Lambda|$, s.t.  $\delta > | \Lambda |$ the corresponding value of tag $t_{\sigma^*}$ is not garbage collected in $\alpha$, at least until execution point $T_2$  in  any of the servers in $S_{\sigma^*}$.
				 
				 Therefore, during the execution fragment between the execution points $T_1$ and $T_2$ of the execution $\alpha$, the tag and coded-element pair is present in the $List$ variable of every in $S_{\sigma^*}$ that is active. As a result, the tag and coded-element pairs, $(t_{\sigma^*}, \Phi_s(v_{\sigma^*}))$ exists in the $List$ received from any
				  $s \in S_{\sigma^*} \cap S_{\pi}$ during operation $\pi$. Note that since $|S_{\sigma^*}| = |S_{\pi}| =\left\lceil \frac{n+k}{2} \right \rceil $ hence
				 $| S_{\sigma^*} \cap S_{\pi} | \geq k$ and hence 
				 $t_{\sigma^*} \in Tags_{dec}^{\geq k} $, the set of decodable tag, i.e., the value $v_{\sigma^*}$ can be decoded
				  by $c_{\pi}$ in $\pi$, which demonstrates that $Tags_{dec}^{\geq k}  \neq \emptyset$. Next we want to 
				  argue that 
				  $t_{max}^* = t_{max}^{dec}$ via a contradiction: we assume 
				  $ \max Tags_{*}^{\geq k}  >  \max Tags_{dec}^{\geq k}  $. Now, consider any tag $t$, which  exists due to our assumption,  such that, 
				  $t \in Tags_{*}^{\geq k} $,  $t \not\in Tags_{dec}^{\geq k} $ and $t > t_{max}^{dec}$.
			%	 
				 Let $S^k_{\pi} \subset S$ be any subset of $k$ servers that responds with $t^*_{max}$ in their $List$ variables to $c_{\pi}$. Note that since $k >  n/3$ hence $|S_{\sigma^*} \cap S_{\pi}|  \geq \left\lceil \frac{n+k}{2} \right \rceil +  \left\lceil \frac{n+1}{3} \right \rceil \geq 1$, i.e., $S_{\sigma^*} \cap S_{\pi} \neq \emptyset$. Then $t$ 
				 must be in some servers in $S_{\sigma^*}$ at $T_2$ and since $t > t_{max}^{dec} \geq t_{\sigma^*}$. 
				 Now since $|\Lambda| < \delta$ hence $(t, \bot)$ cannot be in any server at $T_2$  because there are not enough concurrent write operations (i.e., writes in $\Lambda$) to garbage-collect the coded-elements corresponding to tag $t$, which also holds  for tag  $t^{*}_{max}$. In that case, $t$ must be in $Tag_{dec}^{\geq k}$, a contradiction.
%
				\end{proof}
}

\subsection{Reconfiguration Protocol Properties}
\label{sec:safety:recon}
In this section we analyze the properties that we can achieve through our reconfiguration algorithm. 
The first lemma shows that any two configuration sequences have the same configuration identifiers
in the same indexes. 

\begin{lemma}
\label{lem:consconf}
	For any reconfigurer $r$ that invokes an $\act{reconfig}(c)$ action in an execution $\EX$ 
	of the algorithm, If $r$ chooses to install $c$ in index $k$ of its local $r.cseq$ vector, then $r$ invokes 
	the $Cons[k-1].propose(c)$ instance over configuration $r.cseq[k-1].cfg$.
\end{lemma}

\begin{proof}
	It follows directly from the algorithm. 
\end{proof}

\begin{lemma}
	\label{lem:server:monotonic}
	If a server $s$ sets $s.nextC$ to $\tup{c,F}$ at some state $\st$ in an execution $\EX$ 
	of the algorithm, then $s.nextC = \tup{c,F}$ for any state $\st'$ that appears after $\st$ in 
	$\EX$.
\end{lemma}

\begin{proof}
	Notice that a server $s$ updates the $s.nextC$ variable for some specific configuration $c_k$ 
	in a state $st$ if: (i) $s$ did not receive any value for $c_k$ before (and thus $nextC=\bot$), or (ii) $s$ 
	received a tuple $\tup{c,P}$ and before $\st$ received the tuple $\tup{c',F}$. By Observation \ref{obs:consensus}
	$c=c'$ as $s$ updates the $s.nextC$ of the same configuration $c_k$. Once the tuple becomes equal to 
	$\tup{c,F}$ then $s$ does not satisfy the update condition for $c_k$, and hence in any state $\st'$ after $\st$
	it does not change $\tup{c,F}$.
\end{proof}

\begin{lemma}[Configuration Uniqueness]
\label{lem:unique}

	%Let $\st_1$ and $\st_2$ be any two states of an execution $\EX$ of the algorithm,
	%and $\pr, q$ two participating processes. 
	%be the state after the response action of an operation $\op_1$ from process $p$,
	%and $\st_2$ be the state after the first $\act{read-config}$ call of an operation $\op_2$ from $q$.
	For any processes $\pr, q\in \idSet$ and any states $\st_1, \st_2$ in an execution $\EX$, it must hold that 
	$\config{\cvec{\pr}{\st_1}[i]}=\config{\cvec{q}{\st_2}[i]}$,  $\forall i$ s.t. 
	$\config{\cvec{\pr}{\st_1}[i]},\config{\cvec{q}{\st_2}[i]}\neq \bot$.
\end{lemma}

\begin{proof}
	The lemma holds trivially for $\config{\cvec{\pr}{\st_1}[0]}=\config{\cvec{q}{\st_2}[0]}=c_0$. 
	So in the rest of the proof we focus in the case where $i > 0$. Let us assume 
	w.l.o.g. that $\st_1$ appears before $\st_2$ in $\EX$.
	
	According to our algorithm a process $\pr$ sets $\pr.cseq[i].cfg$ to a configuration 
	identifier $c$ in two cases: (i) either it received $c$ as the result of the consensus 
	instance in configuration $\pr.cseq[i-1].cfg$, or (ii) $\pr$ receives $\config{s.nextC} = c$ from 
	a server $s\in\servers{\config{\pr.cseq[i-1]}}$. Note here that (i) is possible only 
	when $\pr$ is a reconfigurer and attempts to install a new configuration. On the 
	other hand (ii) may be executed by any process in any operation that invokes the 
	$\act{read-config}$ action. We are going 
	to proof this lemma by induction on the configuration index. 
	

	\emph{Base case:} The base case of the lemma is when $i=1$. 
	Let us first assume that $p$ and $q$ receive $c_p$ and $c_q$, as the result of the consensus instance at $\pr.cseq[0].cfg$
	and $q.cseq[0].cfg$ respectively. By Lemma \ref{lem:consconf}, since both processes want to install a configuration 
	in $i=1$, then they have to run $Cons[0]$ instance over the configuration stored in their local $cseq[0].cfg$ variable. 
	Since $\pr.cseq[0].cfg=q.cseq[0].cfg=c_0$ then 
	both $Cons[0]$ instances run over the same configuration $c_0$ and according to Observation \ref{obs:consensus}  
	return the same value, say $c_1$. Hence $c_p=c_q=c_1$ and $\pr.cseq[1].cfg=q.cseq[1].cfg=c_1$.
	 
	 Let us examine the case now where $p$ or $q$ 
	assign a configuration $c$ they received from some server $s\in\servers{c_0}$. According to the
	algorithm only the configuration that has been decided by the consensus instance on 
	$c_0$ is propagated to the servers in $\servers{c_0}$. If $c_1$ is the decided configuration, then 
	$\forall s\in\servers{c_0}$ such that $s.nextC(c_0)\neq\bot$, it holds that $s.nextC(C_0) = \tup{c_1,*}$.
	So if $p$ or $q$ set $\pr.cseq[1].cfg$ or $q.cseq[1].cfg$ to some received configuration, then 
	$\pr.cseq[1].cfg = q.cseq[1].cfg = c_1$ in this case as well. 
	
     \emph{Hypothesis:} We assume  that 
	$\cvec{\pr}{\st_1}[k]=\cvec{q}{\st_2}[k]$  for some $k$, $k \geq 1$.
	
	%\noindent{\bf Induction Hypothesis:} 
	\emph{Induction Step:}  We need to show that the lemma holds for $i=k+1$.
	If both processes retrieve $\config{\pr.cseq[k+1]}$ and $\config{q.cseq[k+1]}$ through consensus, 
	then both $\pr$ and $q$ run consensus
	over the previous configuration. Since according to our hypothesis 
	$\cvec{\pr}{\st_1}[k]=\cvec{q}{\st_2}[k]$ then both process will receive the same
	decided value, say $c_{k+1}$, and hence $\pr.cseq[k+1].cfg=q.cseq[k+1].cfg=c_{k+1}$. Similar to the base case,
	a server in $\servers{c_k}$ only receives the configuration $c_{k+1}$ decided by the consensus instance run over $c_k$. 
	So processes 
	$\pr$ and $q$ can only receive $c_{k+1}$ from some server in $\servers{c_k}$ 
	%even if the processes update their $\pr.cseq[k+1].cfg$ or $q.cseq[k+1].cfg$ with a received 
	%configuration that will be equal to 
	so they can only assign $\pr.cseq[k+1].cfg=q.cseq[k+1].cfg=c_{k+1}$ at Line \ref{algo:reconfigurer}:\ref{line:readconfig:assign}.
	That completes the proof. 
\end{proof}


Lemma \ref{lem:unique} showed that any two operations store the same
configuration in any cell $k$ of their $cseq$ variable. It is not known however 
if the two processes discover the same number of configuration ids. In the following
lemmas we will show that if a process learns about a configuration in a cell $k$ 
then it also learns about all configuration ids for every index $i$, such that $0\leq i\leq k-1$.

\begin{lemma}
\label{lem:confmonotonic}
	In any execution $\EX$ of the algorithm , If for any process $\pr\in\idSet$, $\cvec{\pr}{\st}[i]\neq\bot$ in some state $\st$ in $\EX$,
	then $\cvec{\pr}{\st'}[i]\neq\bot$ in any state $\st'$ that appears after $\st$ in $\EX$. 
\end{lemma}

\begin{proof}
	A value is assigned to $\cvec{\pr}{*}[i]$ either after the invocation of a consensus instance, or while executing
	the $\act{read-config}$ action. Since any configuration proposed for installation cannot be $\bot$ (A\ref{algo:reconfigurer}:\ref{line:install:valid}), 
	and since there is at least one configuration proposed in the consensus instance (the one from $\pr$), then by the validity of the consensus
	service the decision will be a configuration $c\neq\bot$. Thus, in this case $\cvec{\pr}{*}[i]$ cannot be $\bot$.
	Also in the $\act{read-config}$ procedure, $\cvec{\pr}{*}[i]$ is assigned to a value different than $\bot$ according
	to Line A\ref{algo:reconfigurer}:L\ref{line:readconfig:assign}. Hence, if $\cvec{\pr}{\st}[i]\neq\bot$ at state $\st$ 
	then it cannot become $\bot$ in any state $\st'$ after $\st$ in execution $\EX$.
\end{proof}


\begin{lemma}
\label{lem:nogaps}
	Let $\st_1$ be some state in an execution $\EX$ of the algorithm. Then for any 
	process $\pr$, if $k = max\{i: \cvec{\pr}{\st_1}[i]\neq \bot\}$, then 
	$\cvec{\pr}{\st_1}[j]\neq \bot$, for $0\leq j < k$.
\end{lemma}
\begin{proof}
	Let us assume to derive contradiction that there exists $j < k$ such that 
	$\cvec{\pr}{\st_1}[j]=\bot$ and $\cvec{\pr}{\st_1}[j+1]\neq\bot$.
	Suppose w.l.o.g. that $j = k-1$ and that $\st_1$ is the state immediately 
	after the assignment of a value to $\cvec{\pr}{\st_1}[k]$, say $c_k$. 
	Since $\cvec{\pr}{\st_1}[k]\neq\bot$, then $\pr$ assigned $c_k$ to $\cvec{\pr}{\st_1}[k]$ 
	in one of the following cases: 
	(i) $c_k$ was the result of the consensus instance, or
	(ii) $\pr$ received $c_k$ from a server during a $\act{read-config}$ action.
	The first case is trivially impossible as according to Lemma \ref{lem:consconf} 
	$\pr$ decides for $k$ when it runs consensus over configuration $\config{\cvec{\pr}{\st_1}[k-1]}$. 
	Since this is equal to $\bot$, then we cannot run consensus over a non-existent set of 
	processes. 	In the second case, $\pr$ assigns $\cvec{\pr}{\st_1}[k] = c_k$  in Line A\ref{algo:parser}:\ref{line:readconfig:assign}.
	The value $c_k$ was however obtained when $\pr$ invoked $\act{get-next-config}$ on 
	configuration $\config{\cvec{\pr}{\st_1}[k-1]}$. In that action, $\pr$ sends {\sc read-config}
	messages to the servers in $\servers{\config{\cvec{\pr}{\st_1}[k-1]}}$ and waits until a quorum 
	of servers replies. Since we assigned $\cvec{\pr}{\st_1}[k] = c_k$ it means that $\act{get-next-config}$
	terminated at some state $\st'$ before $\st_1$ in $\EX$, and thus: 
	(a) a quorum of servers in $\servers{\config{\cvec{\pr}{\st'}[k-1]}}$
	replied, and (b) there exists a server $s$ among those that replied with $c_k$. 
	According to our assumption however, $\cvec{\pr}{\st_1}[k-1] = \bot$ at $\st_1$. 
	So if state $\st'$ is before $\st_1$ in $\EX$, %is the state after the response step of $\act{get-next-config}$, 
	then by Lemma \ref{lem:confmonotonic}, it follows that $\cvec{\pr}{\st'}[k-1] = \bot$. This however 
	implies that $\pr$ communicated with an empty configuration, and thus no server replied to $\pr$. 
	This however contradicts the assumption that a server replied with $c_k$ to $\pr$. 
	
	Since any process traverses the configuration sequence starting from the initial 
	configuration $c_0$, then with a simple induction we can show that 
	$\cvec{\pr}{\st_1}[j]\neq \bot$, for $0\leq j\leq k$.
\end{proof}

We can now move to an important lemma that shows that any \act{read-config} action 
returns an extension of the configuration sequence returned by any previous \act{read-config} action. 
First, we show that the last finalized configuration observed by any \act{read-config} action is at least as 
recent as the finalized configuration observed by any subsequent \act{read-config} action. 
%Using this lemma we will then show that when 

\begin{lemma}
	\label{lem:config:propagation}
	If at a state $\st$ of an execution $\EX$ of the algorithm, if $\mu(\cvec{\pr}{\st}) = k$ % = \max\{i: \cvec{\pr}{\st}[i]\neq\bot\}$
	for some process $\pr$, then for any element $0\leq j < k$, $\exists Q\in \quorums{\config{\cvec{\pr}{\st}[j]}}$
	such that $\forall s\in Q, s.nextC(\config{\cvec{\pr}{\st}[j]})= \cvec{\pr}{\st}[j+1]$. 
\end{lemma}
\begin{proof}
	This lemma follows directly from the algorithm. Notice that whenever a process assigns a value to 
	an element of its local configuration (Lines  A\ref{algo:parser}:\ref{line:readconfig:assign} and 
	A\ref{algo:reconfigurer}:\ref{line:addconfig:assign}), it then propagates this value to a quorum of the 
	previous configuration (Lines  A\ref{algo:parser}:\ref{line:readconfig:put} and 
	A\ref{algo:reconfigurer}:\ref{line:addconfig:put}). So if a process $\pr$ assigned $c_j$ to an 
	element $\cvec{\pr}{\st'}[j]$ in some state $\st'$ in $\EX$, then $\pr$ may assign a value 
	to the $j+1$ element of $\cvec{\pr}{\st''}[j+1]$ only after $\act{put-config}(\config{\cvec{\pr}{\st'}[j-1]},\cvec{\pr}{\st'}[j])$
	occurs. During $\act{put-config}$ action, $\pr$ propagates $\cvec{\pr}{\st'}[j]$ in a quorum 
	$Q\in\quorums{\config{\cvec{\pr}{\st'}[j-1]}}$. Hence, if $\cvec{\pr}{\st}[k]\neq\bot$, then $\pr$ 
	propagated each $\cvec{\pr}{\st'}[j]$, for $0<j\leq k$ to a quorum of servers $Q\in\quorums{\config{\cvec{\pr}{\st'}[j-1]}}$.
	And this completes the proof. 
%	
%	Let examine a single element $\cvec{\pr}{\st'}[j]$ that is added in the configuration vector at some state $\st'$ 
%	that appears before $\st$ in $\EX$. Whenever a configuration is assigned a process $\pr$ invokes $\act{put-config}$
%	action before adding the next configuration. According to this action, the process sends the discovered configuration
%	and sends it to a quorum of servers. As this process is repeated for every element inserted then the lemma follows. 
\end{proof}


\begin{lemma}[Configuration Prefix]
	\label{lem:prefix}
Let $\op_1$ and $\op_2$ two 
%read/write/install operations 
completed \act{read-config} actions invoked by processes $\pr_1, \pr_2\in\idSet$ 
respectively, such that $\op_1\bef\op_2$ in an execution $\EX$. Let $\st_1$ be the state after the response 
step of $\op_1$ and $\st_2$ the state after the response step 
%termination of the first $\act{read-config}$ 
of $\op_2$. Then 
$\cvec{\pr_1}{\st_1}\preceq_p\cvec{\pr_2}{\st_2}$.
\end{lemma}

\begin{proof}
	Let $\nu_1 = \nu(\cvec{\pr_1}{\st_1})$ and $\nu_2 = \nu(\cvec{\pr_2}{\st_2})$.
	By Lemma \ref{lem:unique} for any $i$ such that $\cvec{\pr_1}{\st_1}[i]\neq\bot$ and 
	$\cvec{\pr_2}{\st_2}[i]\neq\bot$, then $\config{\cvec{\pr_1}{\st_1}[i]}=\config{\cvec{\pr_2}{\st_2}[i]}$.
	Also from Lemma \ref{lem:nogaps} we know that for $0\leq j\leq \nu_1, \cvec{\pr_1}{\st_1}[j] \neq \bot$, 
	and $0\leq j\leq \nu_2, \cvec{\pr_2}{\st_2}[j] \neq \bot$. So if we can show that $\nu_1\leq \nu_2$ then the lemma follows. 
	
	Let $\mu = \mu(\cvec{\pr_2}{\st'})$ %, or simply $\mu$, 
	be the last finalized element which $\pr_2$ established in the beginning of 
	the $\act{read-config}$ action $\op_2$ (Line A\ref{algo:reconfigurer}:\ref{line:readconfig:final}) at some state $\st'$ before $\st_2$. 
	It is easy to see that $\mu\leq \nu_2$. If $\nu_1 \leq \mu$ then $\nu_1\leq \nu_2$ and 
	the lemma follows. Thus, it remains
	to examine the case where $\mu < \nu_1$. Notice that since $\op_1\bef\op_2$ then $\st_1$ appears before 
	$\st'$ in execution $\EX$. By Lemma \ref{lem:config:propagation}, we know that by $\st_1$, 
	%in all the configurations $\cvec{\pr_1}{\st_1}[j]$, for $0\leq j < \nu_1$,  
	$\exists Q\in\quorums{\config{\cvec{\pr_1}{\st_1}[j]}}$, for $0\leq j < \nu_1$,   such that 
	$\forall s\in Q, s.nextC = \cvec{\pr_1}{\st_1}[j+1]$. Since $\mu < \nu_1$, then it must be the case 
	that $\exists Q\in \quorums{\config{\cvec{\pr_1}{\st_1}[\mu]}}$ such that $\forall s\in Q, s.nextC = \cvec{\pr_1}{\st_1}[\mu+1]$.
	But by Lemma \ref{lem:unique}, we know that $\config{\cvec{\pr_1}{\st_1}[\mu]}= \config{\cvec{\pr_2}{\st'}[\mu]}$. 
	Let $Q'$ be the quorum that replies to the $\act{read-next-config}$ occurred in $\pr_2$, on configuration  $\config{\cvec{\pr_2}{\st'}[\mu]}$.
	By definition $Q\cap Q'\neq \emptyset$, thus there is a server $s\in Q\cap Q'$ that sends $s.nextC = \cvec{\pr_1}{\st_1}[\mu+1]$
	to $\pr_2$ during $\op_2$. Since $\cvec{\pr_1}{\st_1}[\mu+1]\neq\bot$ then $\pr_2$ assigns $\cvec{\pr_2}{*}[\mu+1]=\cvec{\pr_1}{\st_1}[\mu+1]$, and 
	repeats the process in the configuration $\config{\cvec{\pr_2}{*}[\mu+1]}$. Since every configuration $\config{\cvec{\pr_1}{\st_1}[j]}$, 
		for $\mu\leq j<\nu_1$, has a quorum of servers with $s.nextC$, then by a simple induction it can be shown that the process will 
		be repeated for at least $\nu_1-\mu$ iterations, and  every configuration
		$\cvec{\pr_2}{\st''}[j]=\cvec{\pr_1}{\st_1}[j]$, at some state $\st''$ before $\st_2$. 
		Thus, $\cvec{\pr_2}{\st_2}[j]=\cvec{\pr_1}{\st_1}[j]$, for $0\leq j\leq \nu_1$. Hence $\nu_1\leq\nu_2$ and the lemma follows in this case as well. 
\end{proof}

Thus far we focused on the configuration member of each element in $cseq$. As operations do get in account
the \emph{status} of a configuration, i.e. $P$ or $F$, in the next lemma we will examine the relationship of 
the last finalized configuration as detected by two operations. First we present a lemma that shows the 
monotonicity of the finalized configurations.

\begin{lemma}
	\label{lem:final:monotonic}
	Let $\st$ and $\st'$ two states in an execution $\EX$ such that $\st$ appears before $\st'$ in $\EX$.
 	Then for any process $\pr$ must hold that $\mu(\cvec{\pr}{\st})\leq \mu(\cvec{\pr}{\st'})$.  %\red{must refer to a read-config}
\end{lemma}

\begin{proof}
	This lemma follows from the fact that if a configuration $k$ is such that 
	$\status{\cvec{\pr}{\st}[k]}=F$ at a state $\st$, then $\pr$ will start any 
	future $\act{read-config}$ action from a configuration $\config{\cvec{\pr}{\st'}[j]}$
	such that $j\geq k$. But $\config{\cvec{\pr}{\st'}[j]}$ is the last finalized configuration 
	at $\st'$ and hence $\mu(\cvec{\pr}{\st'})\geq \mu(\cvec{\pr}{\st})$.
\end{proof}

\begin{lemma}  [Configuration Progress]
	\label{lem:finalconf}
	Let $\op_1$ and $\op_2$ two 
	%read/write/install operations 
	completed \act{read-config} actions invoked by processes $\pr_1, \pr_2\in\idSet$ 
	respectively, such that $\op_1\bef\op_2$ in an execution $\EX$. 
	Let $\st_1$ be the state after the response 
	step of $\op_1$ and $\st_2$ the state after the response step 
	%after the completion of $\op_1$ and $\st_2$ the state after the termination of the first $\act{read-config}$ 
	of $\op_2$. Then 
	$\mu(\cvec{\pr_1}{\st_1})\leq\mu(\cvec{\pr_2}{\st_2})$.
\end{lemma}

\begin{proof}
	By Lemma \ref{lem:prefix} it follows that $\cvec{\pr_1}{\st_1}$ is a prefix of $\cvec{\pr_2}{\st_2}$.
	Thus, if $\nu_1 = \nu(\cvec{\pr_1}{\st_1})$ and $\nu_2 = \nu(\cvec{\pr_2}{\st_2})$, $\nu_1\leq\nu_2$.
	Let $\mu_1=\mu(\cvec{\pr_1}{\st_1})$, such that $\mu_1\leq\nu_1$, be the last element in $\cvec{\pr_1}{\st_1}$,
	where $\status{\cvec{\pr_1}{\st_1}[\mu_1]} = F$. Let now $\mu_2=\mu(\cvec{\pr_2}{\st'})$, 
	be the last element which $\pr_2$ obtained in Line A\ref{algo:parser}:\ref{line:readconfig:final} 
	during $\op_2$ 
	%of the  $\act{read-config}$ action 
	such that $\status{\cvec{\pr_2}{\st'}[\mu_2]} = F$ in some state $\st'$ before $\st_2$. 
	If $\mu_2\geq\mu_1$, and since $\st_2$ is after $\st'$, then by Lemma \ref{lem:final:monotonic} 
	$\mu_2\leq \mu(\cvec{\pr_2}{\st_2})$ and hence $\mu_1\leq \mu(\cvec{\pr_2}{\st_2})$ as well. 
	
	It remains to examine the case where $\mu_2<\mu_1$. Process  $\pr_1$ 
	sets the status of $\cvec{\pr_1}{\st_1}[\mu_1]$ to $F$ in two cases: (i) either when finalizing 
	a reconfiguration, or (ii) when receiving an $s.nextC = \tup{\config{\cvec{\pr_1}{\st_1}[\mu_1]}, F}$ %with status $F$ 
	from some server $s$ during a $\act{read-config}$ action. In both cases $\pr_1$ propagates the 
	$\tup{\config{\cvec{\pr_1}{\st_1}[\mu_1]}, F}$ to a quorum of servers in  
	$\config{\cvec{\pr_1}{\st_1}[\mu_1-1]}$ before completing. We know by Lemma
	\ref{lem:prefix} that since $\op_1\bef\op_2$ then $\cvec{\pr_1}{\st_1}$ is a prefix 
	in terms of configurations of the $\cvec{\pr_2}{\st_2}$. So it must be the case 
	that $\mu_2 < \mu_1 \leq \nu(\cvec{\pr_2}{\st_2})$. Thus, during $\op_2$, %the $\act{read-config}$ action, 
	$\pr_2$ starts from the configuration at index $\mu_2$ and in some iteration 
	performs $\act{get-next-config}$ in configuration $\cvec{\pr_2}{\st_2}[\mu_1-1]$. 
	According to Lemma \ref{lem:unique}, $\config{\cvec{\pr_1}{\st_1}[\mu_1-1]} = \config{\cvec{\pr_2}{\st_2}[\mu_1-1]}$.
	Since $\op_1$ completed before $\op_2$, then it must be the case that $\st_1$ appears before 
	$\st'$ in $\EX$. However, $\pr_2$ invokes the $\act{get-next-config}$ operation in a state $\st''$
	which is either equal to $\st'$ or appears after $\st'$ in $\EX$. Thus, $\st''$ must appear after $\st_1$ in $\EX$.
	From that it follows that when the $\act{get-next-config}$ is executed by $\pr_2$ there is already 
	a quorum of servers in $\config{\cvec{\pr_2}{\st_2}[\mu_1-1]}$, say $Q_1$, that received 
	$\tup{\config{\cvec{\pr_1}{\st_1}[\mu_1]}, F}$from $\pr_1$. 
	Since, $\pr_2$ waits from replies from a quorum of servers from the same configuration, say $Q_2$, and since the 
	$nextC$ variable at each server is monotonic (Lemma \ref{lem:server:monotonic}), then there is a server $s\in \quo{1}\cap \quo{2}$, 
	such that $s$ replies to $\pr_2$ with $s.nextC = \tup{\config{\cvec{\pr_1}{\st_1}[\mu_1]}, F}$. So, 
	$\cvec{\pr_2}{\st_2}[\mu_1]$ gets $\tup{\config{\cvec{\pr_1}{\st_1}[\mu_1]}, F}$, and 
	hence $\mu(\cvec{\pr_2}{\st_2})\geq \mu_1$ in this case as well. This completes our proof.
\end{proof}

\begin{theorem}
	Let $\op_1$ and $\op_2$ two 
%read/write/install operations 
completed \act{read-config} actions invoked by processes $\pr_1, \pr_2\in\idSet$ 
respectively, such that $\op_1\bef\op_2$ in an execution $\EX$. 
Let $\st_1$ be the state after the response 
step of $\op_1$ and $\st_2$ the state after the response step 
%after the completion of $\op_1$ and $\st_2$ the state after the termination of the first $\act{read-config}$ 
of $\op_2$.
%$\cvec{\pr_1}{\st_1}$ and $\cvec{\pr_2}{\st_2}$ at two states $\st_1$ and $\st_2$ 
%in an execution $\EX$ of the algorithm  such that $\st_1$ appears before $\st_2$ in $\EX$. 
Then the following properties hold: 
\begin{enumerate}
\item [$(a)$] 
$\cvec{\pr_2}{\st_2}[i].cfg = \cvec{\pr_1}{\st_1}[i].cfg$,  for $ 1 \leq i \leq \nu(\cvec{\pr_1}{\st_1})$,
\item [$(b)$]
 %$(b)$  
 $\cvec{\pr_1}{\st_1}  \preceq_p \cvec{\pr_2}{\st_2}$, and
\item [$(c)$] 
%$(c)$  
  $\mu(\cvec{\pr_1}{\st_1}) \leq \mu(\cvec{\pr_2}{\st_2})$
  %; and 
%\item [$(d)$]  
%\nn{????$(d)$  $\cvec{\pr_2}{\st_2}[i]   = \cvec{\pr_1}{\st_1}[i]$,  for  $ 1 \leq i \leq \mu(\cvec{\pr_1}{\st_1})$????} 
\end{enumerate}
\end{theorem}

\begin{proof}
Statements $(a)$, $(b)$ and $(c)$ follow from Lemmas \ref{lem:unique}, \ref{lem:prefix}, and 
\ref{lem:final:monotonic}.
%are from  the atomic read-modify-write property of ${\recBox}.\act{add-config}(\cdot)$ 
%and $(d)$ is due to the recon client.
\end{proof}

\subsection{Atomicity  Property of \ares{}}
\label{sec:safety:atomic}
The correctness of \ares{} highly depends on the way the configuration 
sequence is constructed at each client process. Let $\cvec{p}{\state}$ 
denote the configuration sequence $cseq$ at process $p$ in a state $\state$ and 
$\mu(\cvec{\pr}{\state})$ the index of the last finalized configuration in $\cvec{p}{\state}$. 
Then the following properties are preserved by the reconfiguration service:

\begin{theorem}
	Let $\op_1$ and $\op_2$ two 
	%read/write/install operations 
	completed \act{read-config} actions invoked by processes $\pr_1, \pr_2\in\idSet$ 
	respectively, such that $\op_1\bef\op_2$ in an execution $\EX$. 
	Let $\state_1$ the state after the response 
	step of $\op_1$ and $\state_2$ the state after the response step 
	%after the completion of $\op_1$ and $\state_2$ the state after the termination of the first $\act{read-config}$ 
	of $\op_2$.
	%$\cvec{\pr_1}{\state_1}$ and $\cvec{\pr_2}{\state_2}$ at two states $\state_1$ and $\state_2$ 
	%in an execution $\EX$ of the algorithm  such that $\state_1$ appears before $\state_2$ in $\EX$. 
	Then the following properties hold: 
	%\begin{enumerate}
		%\item [$(a)$] 
		$(a)$ \textbf{Configuration Consistency}: 
		$\cvec{\pr_2}{\state_2}[i].cfg = \cvec{\pr_1}{\state_1}[i].cfg$,  for $ 1 \leq i \leq |\cvec{\pr_1}{\state_1}|$,
		%\item [$(b)$]
		$(b)$  
		\textbf{Seq. Prefix}: $\cvec{\pr_1}{\state_1}  \preceq_p \cvec{\pr_2}{\state_2}$, and
		%\item [$(c)$] 
		$(c)$  
		\textbf{Seq. Progress}: $\mu(\cvec{\pr_1}{\state_1}) \leq \mu(\cvec{\pr_2}{\state_2})$
		%; and 
		%\item [$(d)$]  
		%\nn{????$(d)$  $\cvec{\pr_2}{\state_2}[i]   = \cvec{\pr_1}{\state_1}[i]$,  for  $ 1 \leq i \leq \mu(\cvec{\pr_1}{\state_1})$????} 
	%\end{enumerate}
\end{theorem}

Given the properties satisfied by the reconfiguration algorithm of \ares{} 
and assuming that the DAP used satisfy Property~\ref{property:dap}, as presented
in Section \ref{ssec:dap}, then  we have the following result. 

\begin{theorem}[Atomicity]
	In  any execution $\EX$ of \ares{}, if in every configuration $c\in\gseq$,
	$\dagetdata{c}$, $\daputdata{c}{}$, and $\dagettag{c}$
	 %the DAP primitives  
	 satisfy Property~\ref{property:dap}, then ~\ares{} satisfy atomicity.
	%, given that the 
	%$\act{get-data}$, $\act{get-tag}$, and $\act{put-data}$ primitives used satisfy properties
	%\textbf{C1} and \textbf{C2} of Definition \ref{def:consistency}.
\end{theorem}

%In  \ares{},  each configuration 
%may implement the DAPs in a different way as stated below:

\begin{remark}
	Algorithm \ares{} satisfies atomicity even when the implementaton of the  DAPs in two 
	different configurations $c_1$ and $c_2$ are not the same, given that the $c_i.\act{get-tag}$,
	$c_i.\act{get-data}$, and the $c_i.\act{put-data}$ primitives 
	in each $c_i$ satisfy Property~\ref{property:dap}.  
\end{remark}
