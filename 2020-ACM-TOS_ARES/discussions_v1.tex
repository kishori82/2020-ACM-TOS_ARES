 %For a read/write algorithm
%that uses the presented data-primitives, it can provide atomic guarantees if the data primitives 
%satisfy the following consistency properties: 
%To implement an algorithm, which is based on the DAP primitives, the DAP primitives must satisfy the following
%consistency properties.


In this section, we argue that  various implementations of DAPs  can be used in \ares{}. In fact, via \act{reconfig} 
operations, one can implement a highly adaptive 
%strongly consistent 
atomic DSS: 
replication-based can be transformed into erasure-code based DSS; increase or decrease the number of storage servers; 
 study 
the performance of the DSS under various code parameters, etc. 
The insight to implementing various DAPs comes from the observation that the  simple  algorithmic 
	template $A$ (see Alg.~\ref{algo:atomicity:generic1}) for reads and writes protocol combined with 
	any implementation of DAPs, satisfying Property~\ref{property:dap} gives rise to a MWMR atomic memory service.
	Moreover, the read and writes operations terminate as long as the implemented DAPs complete. 
%	implements an atomic  storage if the implementation of the  DAPs satisfy Property \ref{property:dap}, and $A$ preserves liveness (termination),
%	if every invocation of the used DAPs terminate.
\begin{algorithm}[!ht]
	\begin{algorithmic}[2]
		\begin{multicols}{2}
			{\scriptsize
				%\Part{Generic Algorithm $A_1$}
				\Operation{read}{} 
				%\State $wCounter\gets wCounter+1$
				\State $\tup{t, v} \gets \dagetdata{c}$
				\State $\daputdata{c}{ \tup{t,v}}$
				\State return $ \tup{t,v}$
				\EndOperation
				%\Statex
				\Operation{write}{$v$} 
				%\State $wCounter\gets wCounter+1$
				\State $t \gets \dagettag{c}$
				\State $t_w \gets inc(t)$
				\State $\daputdata{c}{\tup{t_w,v}}$
				\EndOperation
				%\EndPart
			}
		\end{multicols}
	\end{algorithmic}
	\caption{Template $A$ for the client-side read/write steps.}
	\label{algo:atomicity:generic1}
	\vspace{-1em}
\end{algorithm}	
	

	A read operation in $A$ performs $\dagetdata{c}$ to retrieve a tag-value pair, $\tup{\tg{},v}$ from a configuration $c$, and then 
	it performs a $c.\act{put-data}(\tup{\tg{},v})$ to propagate that pair to the configuration $c$. A write operation is similar to the read but before 
	performing the $\act{put-data}$ action it generates a new tag which associates with the value to be written.
	%
The following result shows 
	%\cite{ARES:Arxiv:2018} 
	that $A$ is atomic and live,
	if the DAPs satisfy Property~\ref{property:dap} and live.
	 
 \begin{theorem}[\nn{Atomicity} of template $A$]\label{atomicity:A1}
 Suppose the DAP implementation satisfies the consistency properties $C1$ and $C2$ of  Property \ref{def:consistency}
 for a configuration $c\in\confSet$. 
 Then any execution $\EX$  of \nn{algorithm $A$ in configuration $c$}
 %the  atomicity protocols $A_1$  on a 
 %fixed set of servers $S$, of some 
 %configuration $c\in\confSet$,  
 %as in Fig. ~\ref{algo:atomicity:generic1} satisfies atomic read and write and is live if the  primitive functions are live in $\xi$.
 is atomic and live if each DAP \nn{invocation terminates} in $\EX$ \nn{under the failure model $c.\mathcal{F}$}.
 \end{theorem}
 \begin{proof}
 We  prove the atomicity by proving properties $A1$, $A2$ and $A3$ presented in Section \ref{model} for any execution of the algorithm.
					
	\emph{Property $A1$}: Consider two operations $\phi$ and $\pi$ such that $\phi$ completes before $\pi$ is invoked. 
	We need to show that it cannot be  the case that $\pi \prec \phi$. We break our analysis into the following four cases:

	Case $(a)$: {\em Both $\phi$ and $\pi$ are writes}. The $\daputdata{c}{*}$ of $\phi$ completes before 
	$\pi$ is invoked. 
	%which implies that by well-formedness 
	By property {\bf C1} the tag $\tg{\pi}$ returned by the $\dagetdata{c}$ at $\pi$ is 
	at least as large as $\tg{\phi}$. Now, 
	%since $\tg{\pi}$ is larger than $t_{\phi}$, by the steps of 
	since $\tg{\pi}$ is incremented by the write operation then $\pi$ puts a tag $\tg{\pi}'$ such that
	$\tg{\phi} < \tg{\pi}'$ and hence we cannot have $\pi \prec \phi$.
	
	Case $(b)$: {\em $\phi$ is a write and  $\pi$ is a read}. In execution $\EX$ since 
$\daputdata{c} {\tup{t_{\phi}, *}}$ of $\phi$ completes 
	before the $\dagetdata{c}$ of $\pi$ is invoked, by 
	%the well-formedness 
	property {\bf C1} the tag $\tg{\pi}$ obtained from the above
	$\dagetdata{c}$ is at least as large as $\tg{\phi}$. Now $\tg{\phi} \leq \tg{\pi}$ implies that we cannot have $\pi \prec \phi$.
	
	Case $(c)$: {\em $\phi$ is a read and  $\pi$ is a write}.  Let the id of the writer that invokes $\pi$ we $w_{\pi}$.  
	The 
$\daputdata{c}{\tup{\tg{\phi}, *}}$  call of $\phi$ completes
	before  $\dagettag{c}$ of $\pi$ is initiated. Therefore, by 
	%the well-formedness 
	property {\bf C1} %of data-primitives the above 
	$\act{get-tag}(c)$ returns $\tg{}$ such that, $\tg{\phi} \leq \tg{}$. Since $\tg{\pi}$ is equal to $inc(\tg{})$ 
	by design of the algorithm, hence $\tg{\pi} > \tg{\phi}$ and we cannot have $\pi \prec \phi$.
	
	Case $(d)$: {\em Both $\phi$ and $\pi$ are reads}. In execution $\EX$  
the $\daputdata{c}{\tup{t_{\phi}, *}}$ is executed as a part of $\phi$ and 
	completes before $\dagetdata{c}$ is called in $\pi$. By 
	%the well-formedness
	 property {\bf C1} of the data-primitives, 
	we have $\tg{\phi} \leq \tg{\pi}$ and hence we cannot have $\pi \prec \phi$.
	
	\emph{Property $A2$}: Note that because the tag set $\tsSet$ is well-ordered we can show that A2 holds by first showing that
	every write has a unique tag. This means that any two pair of writes can be ordered. 
	Note that 
	a read can be ordered w.r.t. any write operation trivially if the respective tags are different, and by definition, if the 
	tags are equal the write is ordered before the read.
	
	Observe that two tags generated from different writers are necessarily distinct because of the 
	id component of the tag. Now if the operations, say $\phi$ and $\pi$ are writes  from the same writer then, by 
	well-formedness property, the second operation will witness a higher integer part in the tag 
	%is invoked after the first completes, say without loss of generality $\phi$ completes before 
	%$\pi$ is invoked.   In that case the integer part of the tag of $\pi$ is higher 
	%because the well-formedness 
	by property  {\bf C1}, and since the $\dagettag{c}$  is followed by $\daputdata{c}{*}$. Hence $\pi$ is ordered after $\phi$. 
	
	\emph{Property $A3$}:  By {\bf C2} the $\dagetdata{c}$ may return a tag $\tg{}$, only 
	when there exists an operation $\op$ that invoked a $\daputdata{c}{\tup{\tg{},*}}$. Otherwise it returns the initial value. Since a write is the only operation to put a new tag $\tg{}$ in the system then Property $A3$ follows from {\bf C2}.
 \end{proof}
	
	\subsection{Representing  Known Algorithms in terms of data-access primitives}
\label{ssec:generic}

A number of known tag-based algorithms that implement atomic read/write objects 
	(e.g., ABD \cite{ABD96}, \fast \cite{CDGL04} ), can be expressed 
	in terms of DAP. In this subsection we demonstrate how we can transform the very celebrated ABD algorithm \cite{ABD96}.

\paragraph{\mwABD{} Algorithm.}
The multi-writer version of the ABD can be transformed to the generic algorithm Template $A$. Algorithm \ref{algo:dap:abd}
illustrates the three DAP for the ABD algorithm. The $\act{get-data}$ primitive encapsulates the query phase 
of \mwABD{}, while the $\act{put-data}$ primitive encapsulates the propagation phase of the algorithm. 
%Thus, given a configuration $c\in\confSet$ the DAPs in the transformation of ABD are implemented as follows:

\begin{algorithm}[!h]
	%\hrule \F
	\begin{algorithmic}[2]
		\begin{multicols}{2}{\scriptsize
				\Part{Data-Access Primitives at process $\pr$}
				\Procedure{c.put-data}{$\tup{\tg{},v})$}
				\State {\bf send} $(\text{{\sc write}}, \tup{\tg{},v})$ to each $s \in \servers{c}$
				\State {\bf until} $\exists \quo,  \quo \in \quorums{c}$ s.t. $\pr$ receives {\sc ack} from $\forall s\in\quo{}$
				\EndProcedure
				
				\Statex				
				
				\Procedure{c.get-tag}{}
				%	\State {\bf send} $(\text{\act{query}},\rdr)$ to every server $s\in \bigcup_{cseq[i]}members(\qs_{cseq[i].conf})$
				\State {\bf send} $(\text{{\sc query-tag}})$ to each  $s\in \servers{c}$
				\State {\bf until}    $\exists \quo{}, \quo{}\in\quorums{c}$ s.t. 
				\State\TT$\pr$ receives $\tup{\tg{s},v_s}$ from $\forall s\in\quo{}$ 
				\State $\tg{max} \gets \max(\{\tg{s} : \pr \text{ received } \tup{\tg{s},v_s} \text{ from } s \})$
				\State {\bf return} $\tg{max}$
				\EndProcedure
				
				\Statex
				
				\Procedure{c.get-data}{}
				%	\State {\bf send} $(\text{{\sc query}},\rdr)$ to every server $s\in \bigcup_{cseq[i]}members(\qs_{cseq[i].conf})$
				\State {\bf send} $(\text{{\sc query}})$ to each  $s\in \servers{c}$
				\State {\bf until}    $\exists \quo{}, \quo{}\in\quorums{c}$ s.t. 
				\State\TT $\pr$ receives $\tup{\tg{s},v_s}$ from $\forall s\in\quo{}$ 
				\State $\tg{max} \gets \max(\{\tg{s} : \rdr_i \text{ received } \tup{\tg{s},v_s} \text{ from } s \})$
				\State {\bf return} $\{\tup{\tg{s},v_s}:\tg{s}=\tg{max} \wedge ~\pr \text{ received } \tup{\tg{s},v_s} \text{ from } s\}$
				\EndProcedure
				\EndPart
			}\end{multicols}
				
			\\\hrulefill 
			
			\begin{multicols}{2}{\scriptsize
				\Part{Primitive Handlers at server $s_i$ in configuration $c$}
				\Receive{{\sc query-tag}}{} %\Comment{Called upon reception of a message}
				%\State $\act{handle-get-tag(c_k)}$
				\State send $\tg{}$ to $q$
				\EndReceive
				
				\Statex
				
				\Receive{{\sc query}}{}
				%\State $\act{handle-get-data(c_k)}$
					\State send $\tup{\tg{}, v}$ to $q$
				\EndReceive
				
				\Statex
				\Statex
				
				\Receive{{\sc write}, $\tup{\tg{in}, v_{in}}$}{}
				%\State $\act{handle-put-data(c_k)}$
					\If {$\tg{in}> \tg{}$} 	\label{line:server:tg-comparison}
						\State  $\tup{\tg{},v}\gets \tup{\tg{in},v_{in}}$ \label{line:server:update}
					\EndIf
					\State  send  {\sc ack} to $q$ 	\label{line:server:reply}
				\EndReceive
				\EndPart
				
		}\end{multicols}	
	\end{algorithmic}
	%\hrule \B
	\caption{Implementation of DAP for ABD at each process $\pr$ using configuration $c$}
	\label{algo:dap:abd}
\end{algorithm} 

Let us now examine if the primitives satisfy properties \textbf{C1} and \textbf{C2} of Property \ref{property:dap}.
We begin with a lemma that shows the monotonicity of the tags at each server.

\begin{lemma} 
	\label{lem:server:tag:monotonic}
	Let $\st$ and $\st'$ two states in an execution $\EX$ such that $\st$ appears before $\st'$ in $\EX$.
	Then for any server $s\in\srvSet$ it must hold that $\atT{s.tag}{\st}\leq\atT{s.tag}{\st'}$.
\end{lemma}

\begin{proof}
	According to the algorithm, a server $s$ updates its local tag-value pairs when it receives a message
	with a higher tag. So if $\atT{s.tag}{\st}=\tau$ then in a state $\st'$ that appears after $\st$ in $\EX$,
	$\atT{s.tag}{\st'}\geq \tau$.
\end{proof}

In the following two lemmas we show that property \textbf{C1} is satisfied, that is if a $\act{put-data}$ action completes,
then any subsequent $\act{get-data}$ and $\act{get-tag}$ actions will discover a higher tag than the one propagated 
by that $\act{put-data}$ action.

\begin{lemma}
	\label{lem:putdata:gettag}
	Let $\phi$ be a $\daputdata{c}{\tup{\tau,v}}$ action invoked by $\pr_1$ and $\gamma$ be a $\dagettag{c}$ action
	invoked by $\pr_2$ in a configuration $c$, such that 
	$\phi\bef\gamma$ in an execution $\EX$ of the algorithm. Then $\gamma$ returns a tag $\tau_\gamma \geq \tau$. 
\end{lemma}

\begin{proof}
	The lemma follows from the intersection property of quorums. In particular, during the $\daputdata{c}{\tup{\tau,v}}$ 
	action, $\pr_1$ sends the pair $\tup{\tau,v}$ to all the servers in $\servers{c}$ and waits until all the servers in 
	a quorum $Q_i\in\quorums{c}$ reply. When those replies are received then the action completes. 
	
	During a $\dagetdata{c}$ action on the other hand, $\pr_2$ sends query messages to all the servers in $\servers{c}$ 
	and waits until all servers in a quorum $Q_j\in\quorums{c}$ (not necessarily different than $Q_i$) reply. By definition 
	$Q_i \cap Q_j\neq\emptyset$, thus any server $s\in Q_i\cap Q_j$ reply to both $\phi$ and $\gamma$ actions.  
	By Lemma \ref{lem:server:tag:monotonic} and since $s$ received a tag $\tau$, then $s$ replies to $\pr_2$ with a tag
	$\tau_s\geq\tau$. Since $\gamma$ returns the maximum tag it discovers then $\tau_\gamma \geq \tau_s$. Therefore
	$\tau_\gamma \geq \tau$ and this completes the proof.
\end{proof}

With similar arguments and given that each value is associated with a unique tag then we can show the following lemma.

\begin{lemma}
	\label{lem:putdata:getdata}
	Let $\pi$ be a $\daputdata{c}{\tup{\tau,v}}$ action invoked by $\pr_1$ and $\phi$ be a $\dagetdata{c}$ action
	invoked by $\pr_2$ in a configuration $c$, such that $\pi\bef\phi$ in an execution $\EX$ of the algorithm. 
	Then $\phi$ returns a tag-value  $\tup{\tau_\phi, v_\phi}$
	such that $\tau_\phi \geq \tau$. 
\end{lemma}

Finally we can now show that property \textbf{C2} also holds. 

\begin{lemma}
	If $\phi$ is a $\dagetdata{c}$ that returns $\tup{\tg{\pi}, v_\pi } \in \tsSet \times \valSet$, 
	then there exists $\pi$ such that $\pi$ is a $\daputdata{c}{\tup{\tg{\pi}, v_{\pi}}}$ and $\phi \not \rightarrow \pi$.
\end{lemma}

\begin{proof}
	This follows from the facts that (i) servers set their tag-value pair to a pair received by a $\act{put-data}$ action, and (ii) 
	a $\act{get-data}$ action returns a tag-value pair that it received from a server. So if a $\dagetdata{c}$  operation $\phi$ 
	returns a tag-value pair $\tup{\tg{\pi}, v_\pi }$, there should be a server $s$  that replied to that operation with 
	$\tup{\tg{\pi}, v_\pi }$,  and $s$ received $\tup{\tg{\pi}, v_\pi }$ from some $\daputdata{c}{\tup{\tg{\pi}, v_\pi }}$
	action, $\pi$. Thus, $\pi$ can proceed or be concurrent with $\phi$, and hence $\phi\not\bef \pi$.
\end{proof}

